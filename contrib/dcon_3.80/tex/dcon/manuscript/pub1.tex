\documentclass[prb,twocolumn,showpacs,preprintnumbers,amsmath,amssymb]{revtex4}

\usepackage[dvips]{graphicx}
\usepackage{dcolumn}
\usepackage{bm}

\renewcommand*{\v}[1]{\hbox{\bfseries #1}}
\renewcommand*{\t}[1]{\hbox{\sffamily\bfseries #1}}
\newcommand{\vm}[1]{\mbox{\boldmath$#1$}}

\begin{document}

\title{The direct criterion of Newcomb for the \\ stability of an
axisymmetric toroidal plasma} 

\author{A. H. Glasser}
\email{ahg@lanl.gov}
\affiliation{Mail Stop K717, Los Alamos National Laboratory, 
	P.O. Box 1663, Los Alamos, NM 87545}

\date{\today}

\begin{abstract}
A new method is introduced for determining the magnetohydrodynamic
stability of an axisymmetric toroidal plasma, based on a toroidal
generalization of the method developed by Newcomb for cylindrical
plasmas.  For toroidal mode number $n \ne 0$, the problem is reduced to
the solution of a high-order complex system of ordinary differential
equations, the Euler-Lagrange equation for minimizing the potential
energy, and the evaluation of a real Critical Determinant whose poles
indicate the presence of ideal instabilities.  Coupling to a vacuum
region outside the plasma incorporates the effects of free-boundary
instabilities.  The asymptotic behavior of the solutions in the
neighborhood of singular surfaces provides the necessary information for
coupling to singular layers incorporating resistivity, inertia, and
rotation.  The method lends itself to efficient implementation using
adaptive numerical integration.
\end{abstract}

\pacs{52.30.Cv, 52.55.Fa}

\maketitle

\section{\label{sec:intro}Introduction}

A key step in understanding the behavior of tokamaks is the
determination of the stability of the confined plasma to
magnetohydrodynamic (MHD) perturbations.  Such instabilities are
believed to set limits on the achievable current density and plasma
pressure and to influence the onset of disruptions which can damage the
device.  Stability studies constitute a major effort in interpretation
of experiments and design studies for new tokamaks.

Most existing MHD stability codes\cite{rcg76, rg81, wk76, lcb81, lmd84,
rcg83} are based on a procedure in which the perturbed potential and
kinetic energies are expanded in a set of 2-dimensional basis functions,
reducing the numerical problem to the solution of a large matrix
eigenvalue problem for the growth rates, frequencies, and eigenfunctions
of the entire MHD spectrum.  These methods can suffer from poor
convergence of the basis functions, especially for difficult equilibria
with high $\beta$ (the ratio $2 \mu_0 P/ B^2$ of plasma pressure to
magnetic field energy density), strong noncircularity, and poloidal
divertors.  They require large expenditures of human and computer time.
Much of the effort required by such methods goes into determining
eigenvalues and eigenvectors which may not be needed when the real
concern is simply the location of the stability boundary and perhaps the
distance of a given equilibrium from that boundary.

An alternative procedure was introduced by Newcomb for cylindrical
plasmas.\cite{wan60} He found conditions for the existence of test
functions which satisfy the boundary conditions and make the potential
energy $\delta W$ negative, relying on the well-known result that the
existence of such test functions implies the existence of exponentially
growing perturbations.  His condition involves integrating a real
2nd-order ordinary differential equation (ODE), the Euler-Lagrange
equation for minimizing the potential energy, and determining whether
the solution $\xi$ changes sign between successive singular points.
This condition can be interpreted as determining whether there is a mode
which fits into the ``box'' of the equilibrium.  We refer to it as the
Direct Criterion of Newcomb, or \texttt{DCON}.

The purpose of this paper is to generalize Newcomb's criterion to an
axisymmetric toroidal plasma with toroidal mode number $n \ne 0$.  The
key difference between a cylinder and a torus is that $\theta$, the
poloidal angle which goes around the short way, is no longer a symmetry,
as it is in a cylinder.  As a result, one can no longer treat individual
Fourier components in this variable, because toroidicity and
noncircularity couple different components and thus require that they be
treated simultaneously.  The generalization of Newcomb's 2nd-order real
Euler-Lagrange equation is a complex system of ODEs of order $2M$, where
$M$ is the number of poloidal Fourier components retained in the
treatment.  This system can be solved numerically with a very efficient
adaptive ODE solver.

In order to generalize Newcomb's sign-change criterion, it is necessary
to identify a real scalar quantity whose sign change plays the same role
in the toroidal case as that of Newcomb's displacement $\xi$ in the
cylindrical case.  This quantity has been found as a critical
determinants constructed from the Euler-Lagrange solutions with
different initial conditions.  Its reality is a consequence of the
Hamiltonian symmetry of the Euler-Lagrange equation, which is in turn a
consequence of the well-known self-adjointness of the linearized ideal
MHD equations.  Its role in the theory is demonstrated by a
generalization of Newcomb's use of the Hilbert Invariant Integral.
Another important difference between the cylindrical and toroidal cases
is that in the former, successive intervals between singular surfaces
can be treated independently, while in the latter they are coupled by
nonresonant components which must be continuous across singular
surfaces.

Bineau\cite{mb61} attempted a treatment similar to ours but for general,
non-axisymmetric toroidal plasmas.  His approach has many of the same
features as this one, but it is more formal, it is not clear how to
apply it as a practical stability program, and he does not obtain our
key result, the Critical Determinant.  Section \ref{sec:conclude}
discusses the difficulties of extending our treatment to nonaxisymmetric
systems.  Connor {\it et al.}\cite{jwc88} and Fitzpatrick \textit{et
al.}\cite{rf93} treat the equivalent of our Euler-Lagrange equation in
the context of a large aspect ratio expansion.  Their interest is in
toroidal resistive tearing modes; they do not discuss the symmetries of
the equations or the generalization of Newcomb's criterion for ideal
stability.  Dewar and Pletzer\cite{rld90, ap94} discuss a generalization
of Newcomb's Euler-Lagrange equation to axisymmetric toroidal plasmas,
but they do not express its properties as a coupled system of ODEs for
the complex Fourier amplitudes, nor do they generalize Newcomb's
criterion.  Their concern is with resistive stability, and their
numerical treatment uses an expansion in finite elements in the radial
direction.

The remainder of this paper is organized as follows.  In Section
\ref{sec:euler} we introduce a coordinate system and a vector
representation used to evaluate and minimize the potential energy
$\delta W$, derive the Euler-Lagrange equation, and discuss the
symmetries of this equation.  In Section \ref{sec:sing} we discuss the
behavior of the Euler-Lagrange equation in the neighborhood of its
singular points at the resonant surfaces, the magnetic axis, and the
separatrix.  In Section \ref{sec:fixed} we prove the toroidal
generalization of Newcomb's criterion in terms of the the critical
determinant.  In Section \ref{sec:free} we extend the treatment to
include a vacuum region surrounding the plasma region, enabling us to
treat external instabilities.  In Section \ref{sec:conclude}, we discuss
our conclusions.  Appendix \ref{sec:coef} gives detailed expressions for
the coefficient matrices which appear in the Euler-Lagrange equation.
SI units are used.

\section{\label{sec:euler}The Euler-Lagrange equation}

In this section, we first describe the equilibrium and introduce a
coordinate system and a representation for perturbed vectors.  We then
use these to express the ideal MHD potential energy $\delta W$, and
derive the Euler-Lagrange equation as the condition that $\delta W$ be
stationary with respect to small changes in the perturbation.  We
conclude with a discussion of the symmetry properties of this equation.

We treat a stationary equilibrium\cite{mdk58} satisfying the pressure
balance equation
\[
\v{J} \times \v{B} = \mu_0 \nabla P
\]
where \v{B} is the magnetic field, $\v{J} = \nabla \times \v{B}/\mu_0$
is the plasma current, \v{B} and \v{J} are divergenceless, and $P$ is
the plasma pressure.  We assume that the system is axisymmetric,
\textit{i.e.}  all equilibrium scalars are independent of the azimuthal
angle $\phi$.  With these assumptions, the magnetic field can be
represented as
\[
\v{B} = f \nabla \phi + \frac{\chi'}{2 \pi} \nabla \phi \times \nabla \psi
\]
where $f = R B_T$, $R$ is the radius from the axis of symmetry, $B_T$ is
the toroidal component of the magnetic field, and $\chi$ is the poloidal
magnetic flux, and $P$, $\chi$, and $f$ are constant on a flux surface,
{\it i.e.} a surface of constant $\psi$, with $\psi$ an arbitrary
surface label.  The poloidal flux satisfies the Grad-Shafranov equation,
\[
\Delta^* \chi 
	\equiv R^2 \nabla \cdot \left( \frac{1}{R^2} \nabla \chi \right)
	= - \frac{4 \pi^2}{\chi'} (ff' + \mu_0 R^2 P')
\]

This static equilibrium does not allow for strong plasma rotation,
velocities of order the sound or Alfv\'en velocities, for which inertial
and anisotropic pressure effects can be important.  Such fast rotation
can occur in the presence of strong neutral beam injection.  However, we
do not exclude the case of diamagnetic rotation in which these effects
are small and affect primarily the neighborhood of resonant surfaces.

Location within the region of closed magnetic field lines is described
in terms of straight-fieldline coordinates.  Each flux surface is
labeled by an arbitrary coordinate $\psi$, which increases monitonically
from the axis to the plasma edge.  It is commonly chosen to be linear in
the poloidal flux and normalized to go from 0 at the axis to 1 at the
edge.  Position on a flux surface is described in terms of the periodic
variables $\theta$ and $\zeta$, which increase by 1 (not $2\pi$) after
one turn around the torus the short way and the long way, respectively.
Physical scalars must be single-valued, returning to their original
values after $\theta$ or $\zeta$ increases by one.  The system has
Jacobian
\[
{\cal J} \equiv (\nabla \psi \cdot \nabla \theta \times \nabla \zeta)^{-1}.
\]
The contravariant representation of the equilibrium magnetic field is
given by
\[
\v{B} = \chi' (\nabla \zeta - q \nabla \theta) \times \nabla \psi,
\]
where $q$, the safety factor or winding number. is a function of $\psi$
only.  Derivatives along the magnetic field, which play a key role in
MHD stability theory, have the simple representation
\[
\v{B} \cdot \nabla {\cal F} = \frac{\chi'}{\cal J}
	\left( \frac{\partial}{\partial \theta} 
	+ q \frac{\partial}{\partial \zeta} \right) 
	{\cal F}(\psi,\theta,\zeta)
\]
Here and subsequently, primes denote derivatives with respect to $\psi$.
The poloidal and toroidal currents constant on a flux surface are given
by
\[
\v{J} \cdot \nabla \theta = -\frac{2\pi f'}{\cal J}, \qquad 
\v{J} \cdot \nabla \zeta = q \v{J} \cdot \nabla \theta
	 - \frac{\mu_0 P'}{\chi'}
\]
while the normal current $\v{J} \cdot \nabla \psi$ vanishes.

The perturbed displacement vector \vm{\xi} is described in terms of its
contravariant representation,

\begin{equation}
\vm{\xi} = {\cal J} \left( \xi_\psi \nabla \theta \times \nabla
	\zeta + \xi_\theta \nabla \zeta \times \nabla \psi + \xi_\zeta
	\nabla \psi \times \nabla \theta \right)
\label{eq:xi_vector}
\end{equation}
and a useful combination of these components, the surface displacement
\begin{equation}
\xi_s \equiv {\cal J} \left( \vm{\xi}
	\times \v{B} \right)  
	\cdot \left( \nabla \theta \times \nabla \zeta \right)
= \chi' (q \xi_\theta - \xi_\zeta)
\label{eq:xi_surface}
\end{equation}
Likewise, the perturbed magnetic field is expressed in terms of its
contravariant components,
\begin{eqnarray}
\v{Q}=&&\nabla \times (\vm{\xi} \times \v{B})
	= {\cal J} \left( Q_\psi \nabla \theta \times \nabla \zeta 
	\right. \nonumber \\
+&& \left. Q_\theta \nabla \zeta \times \nabla \psi
	+ Q_\zeta \nabla \psi \times \nabla \theta \right)
\label{eq:q_vector}
\end{eqnarray}
These components are related to the components of \vm{\xi} by
\begin{eqnarray}
Q_\psi =&& \frac{\chi'}{\cal J} \left( 
	\frac{\partial}{\partial \theta}
	+ q \frac{\partial}{\partial \zeta} \right) 
	\xi_\psi \nonumber \\
Q_\theta =&& - \frac{1}{\cal J}
	\frac{\partial}{\partial \psi}
	\left( \chi' \xi_\psi \right)
	+ \frac{1}{\cal J}
	\frac{\partial \xi_s}{\partial \zeta} \nonumber \\
Q_\zeta =&& - \frac{1}{\cal J}
	\frac{\partial}{\partial \psi}
	\left( q \chi' \xi_\psi \right)
	- \frac{1}{\cal J}
	\frac{\partial \xi_s}{\partial \theta} 
\label{eq:q_components}
\end{eqnarray}

The ideal MHD perturbed potential energy is given by\cite{ibb58}
\begin{eqnarray}
\delta W =&& \frac{1}{2 \mu_0} \int d\v{x} 
	\left[ Q^2 + \v{J} \cdot 
	\vm{\xi} 
	\times \v{B} \vm{\xi} 
	\times \v{Q} \right. \nonumber \\
&& \left. + \mu_0 (
	\vm{\xi} 
	\cdot \nabla P) (\nabla \cdot \vm{\xi})
	+ \mu_0 \gamma P (\nabla \cdot 
	\vm{\xi})^2 \right]
\label{eq:dw1}
\end{eqnarray}
When Eqs. (\ref{eq:xi_vector}) - (\ref{eq:q_components}) are introduced
into Eq. (\ref{eq:dw1}), we obtain
\begin{widetext}
\begin{eqnarray}
\delta W = \frac{1}{2 \mu_0} \int d\psi d\theta d\zeta  {\cal J}
	&&\left\{ 
	\left[ \nabla \xi_s \times  \nabla \psi
	+ (\nabla \theta \times \nabla \zeta) 
	{\cal J} \v{B} \cdot \nabla \xi_\psi - \v{B} \xi'_\psi
	- \left[ \chi'' \nabla \zeta \times \nabla \psi
	+ (q \chi')' \nabla \psi \times \nabla \theta \right] 
	\xi_\psi \right]^2 \right. \nonumber \\
&&+ \xi_\psi \v{J} \cdot \nabla \xi_s 
	- \xi_s \v{J} \cdot \nabla \xi_\psi
	+ \left[ J_\theta (q \chi')' - J_\zeta \chi'' \right] \xi_\psi^2
	+ \frac{\mu_0 P'}{\cal J} \left( {\cal J} \xi_\psi^2 \right)'
	\nonumber \\
&&\left. + \frac{1}{\cal J} \frac{\partial}{\partial \theta} 
	\left( {\cal J} \mu_0 P' \xi_\psi \xi_\theta \right)
	+ \frac{1}{\cal J} \frac{\partial}{\partial \zeta} 
	\left( {\cal J} \mu_0 P' \xi_\psi \xi_\zeta \right)
	+ \gamma \mu_0 P (\nabla \cdot 
	\vm{\xi})^2 \right\}.
\end{eqnarray}
\end{widetext}

The first two terms on the last line are perfect derivatives which
vanish on integration over a flux surface.  The last term is
positive-definite and is the only term containing $\xi_\theta$.
Minimization with respect to $\xi_\theta$ eliminates this term for
toroidal mode number $n \ne 0$.  The minimizing perturbations are thus
divergenceless.  This will be assumed henceforth, and the compression
term will be dropped.

The remaining terms contain only the normal displacement $\xi_\psi$ and
the surface displacement $\xi_s$.  While $\xi_\psi$ enters through its
radial derivative $\xi_\psi'$ as well as angular derivatives with
respect to $\theta$ and $\zeta$, $\xi_s$ enters only through its angular
derivatives, which will permit us to eliminate it from the treatment and
reduce $\delta W$ to a form involving only the radial displacement and
its radial derivative, as in Newcomb's treatment of the cylindrical
plasma.

We now introduce Fourier series for the normal and surface
displacements, 
\begin{eqnarray}
\left( \begin{matrix} \xi_s \\ \xi_\psi \end{matrix} \right)
(\psi,\theta,\zeta) &&= \sum_{m=-\infty}^\infty \sum_{n=-\infty}^\infty
\left. \left( \begin{matrix} \bar{\xi}_s \\ \bar{\xi}_\psi \end{matrix} 
	\right) \right|_{m,n} (\psi) \nonumber \\
&& \times \exp\left[ 2 \pi i \left( m \theta - n \zeta \right) \right]
\label{eq:fourier}
\end{eqnarray}
In practice, the infinite sums in Eq. (\ref{eq:fourier}) are truncated to include a
finite number $M$ of components.  The reality of the physical
perturbations implies that the Fourier coefficients satisfy the reality
conditions
\begin{equation}
\left( \begin{matrix} \xi_s \\ \xi_\psi \end{matrix} \right) 
= \left( \begin{matrix} \xi_s \\ \xi_\psi \end{matrix} \right)^*
\Rightarrow \left. \left( \begin{matrix} \bar{\xi}_s \\ \bar{\xi}_\psi 
	\end{matrix} \right) \right|_{-m,-n}
= \left. \left( \begin{matrix} \bar{\xi}_s \\ \bar{\xi}_\psi 
	\end{matrix} \right) \right|_{m,n}^*
\label{eq:define_xis}
\end{equation}
A useful, compact notation is to introduce the complex $M$-vectors of
Fourier components of the radial and normal displacements.
\begin{eqnarray}
\left( \begin{matrix} \Xi_s \\ \Xi_\psi \end{matrix} \right) (\psi) 
\equiv \left\{ \left. 
	\left( \begin{matrix} \bar{\xi}_s \\ \bar{\xi}_\psi 
	\end{matrix} \right) 
	\right|_{m,n}(\psi),\ m_\textrm{low} \le m \le m_\textrm{high} 
	\right\}. \nonumber \\
\label{eq:Xi_vector}
\end{eqnarray}

With these definitions, the potential energy can be expressed as 
\begin{widetext}
\begin{equation}
\delta W = \frac{1}{2 \mu_0} \int d\psi [
	\Xi_s^\dagger \t{A} \Xi_s 
	+ \Xi_s^\dagger (\t{B} \Xi_\psi' + \t{C} \Xi_\psi)
	+ (\Xi_\psi'^\dagger \t{B}^\dagger
	+ \Xi_\psi^\dagger \t{C}^\dagger) \Xi_s
	+ \Xi_\psi'^\dagger \t{D} \Xi_\psi'
	+ \Xi_\psi'^\dagger \t{E} \Xi_\psi
	+ \Xi_\psi^\dagger \t{E}^\dagger \Xi_\psi'
	+ \Xi_\psi^\dagger \t{H} \Xi_\psi ],
\label{eq:dw3}
\end{equation}
\end{widetext}
where \t{A}, \t{B}, \t{C}, \t{D}, \t{E}, and \t{H} are complex $M \times
M$ matrices, dagger denotes Hermitian conjugate, and \t{A}, \t{D}, and
\t{H} are self-adjoint.  Detailed expressions for these matrices are
given in Eq. (\ref{eq:define_a-h}) of Appendix \ref{sec:coef}.

Since $\Xi_s$ enters Eq. (\ref{eq:dw3}) only algebraically, and since
\t{A} is found to be everywhere nonsingular, $\delta W$ can be minimized
with respect to $\Xi$ simply by solving the matrix equation
\begin{eqnarray}
&&\t{A} \Xi_s + \t{B} \Xi_\psi' + \t{C} \Xi_\psi = 0, \nonumber \\
&&\Xi_s = - \t{A}^{-1} (\t{B} \Xi_\psi' + \t{C} \Xi_\psi)
\end{eqnarray}
When this expression is substituted for $\Xi_s$ in Eq. (\ref{eq:dw3}),
we obtain a form of $\delta W$ involving only $\Xi_\psi$ and its radial
derivative,
\begin{equation}
\delta W = \int d\psi \left[
\Xi_\psi'^\dagger \t{F} \Xi_\psi'
+ \Xi_\psi'^\dagger \t{K} \Xi_\psi
+ \Xi_\psi^\dagger \t{K}^\dagger \Xi_\psi'
+ \Xi_\psi^\dagger \t{G} \Xi_\psi \right]
\label{eq:dw4}
\end{equation}
in terms of the composite matrices
\begin{eqnarray}
\t{F} &&\equiv \t{D} - \t{B}^\dagger \t{A}^{-1} \t{B} = \t{F}^\dagger, \nonumber \\
\t{K} &&\equiv \t{E} - \t{B}^\dagger \t{A}^{-1} \t{C} \ne \t{K}^\dagger. \nonumber \\
\t{G} &&\equiv \t{H} - \t{C}^\dagger \t{A}^{-1} \t{C} = \t{G}^\dagger, 
\label{eq:define_fkg}
\end{eqnarray}
Simplified expressions for \t{F} and \t{K} are given in
Eqs. (\ref{eq:simplify_f}) and (\ref{eq:simplify_k}) of Appendix
\ref{sec:coef}.

The Euler-Lagrange equation is the condition that $\delta W$ be
stationary with respect to small changes in $\Xi_V(V)$,
\begin{equation}
\left( \t{F} \Xi_\psi' + \t{K} \Xi_\psi \right)' - \left( \t{K}^\dagger
\Xi_\psi' + \t{G} \Xi_\psi \right) = 0
\label{eq:euler1}
\end{equation}
a system of $M$ coupled 2nd-order ODEs.  This is a necessary but not
sufficient condition that $\delta W$ be a minimum.  Whether the
stationary point is a minimum depends on additional considerations
discussed in Section IV.  However, these conditions are expressed in
terms of the properties of the solutions to Eq. (\ref{eq:euler1}), which
therefore plays a central role in the theory.

The relationship to Newcomb's Euler-Lagrange equation is now
straightforward.  For the cylindrical case, the Fourier components
decouple.  The matrices \t{F}, \t{G}, and \t{K} become diagonal and can
be regarded as real scalars.  In this case, the term $(\t{K} \Xi_\psi)'$
can be expanded, the contribution from $\Xi'$ cancels the other term in
the equation, and the contribution from $\t{K}'$ can be combined with
\t{G} to give Newcomb's equation.  In the more general toroidal case,
\t{K} is not self-adjoint, and no real simplification is obtained by
this procedure.

It is useful, for both analytical and numerical purposes, to express
Eq. (\ref{eq:euler1}) as an equivalent system of $2M$ coupled 1st-order
ODEs.  This can be done by introducing the complex $2M$-vector
\begin{equation}
\v{u} \equiv \left( \begin{matrix} \Xi_\psi \\ \t{F} \Xi_\psi' + \t{K}
\Xi_\psi \end{matrix} \right)
\label{eq:define_u}
\end{equation}
and the complex $2M \times 2M$ matrix
\begin{equation}
\t{L} \equiv \left( \begin{matrix}
- \t{F}^{-1} \t{K} & \t{F}^{-1} \\
\t{G} - \t{K}^\dagger \t{F}^{-1} \t{K} & \t{K}^\dagger \t{F}^{-1}
\end{matrix} \right). 
\label{eq:define_L}
\end{equation}
in terms of which Eq. (\ref{eq:euler1}) can be written
\begin{equation}
\v{u}' = \t{L} \v{u}.
\label{eq:euler2}
\end{equation}

The matrix \t{L} has an important symmetry.  The off-diagonal blocks are
self-adjoint, while the diagonal blocks are the negative of each other's
adjoint.  More formally, in terms the unit symplectic matrix
\[
\t{J} \equiv \left( \begin{matrix}
\t{0} & \t{I} \\ -\t{I} & \t{0} \end{matrix} \right)
\]
\t{L} has the Hamiltonian property, $\t{J} \t{L} \t{J} = \t{L}^\dagger$.
In Hamiltonian terms, the second $M$ components of \v{u} in
Eq. (\ref{eq:define_u}) can be regarded as the conjugate momenta of the
first $M$ components, the displacements.  This is understood in abstract
terms, since the independent variable $\psi$ is not time, and the
variables are complex Fourier components.  The Hamiltonian symmetry of
\t{L} may be thought of as a manifestation in this representation of the
well-known self-adjointness of ideal MHD.

Corresponding to the Hamiltonian symmetry of \t{L}, the fundamental
matrix of solutions has the property of being symplectic.  Equation
(\ref{eq:euler2}) is a $2M$th-order ODE and therefore has $2M$
independent solutions, corresponding to different initial conditions.
These $2M$ column vectors can be assembled into a $2M \times 2M$ matrix
\t{U} which also satisfies Eq. (\ref{eq:euler2}), with \v{u} replaced by
\t{U}.  \t{U} is a linear canonical transformation, the propagator of
the system.  Using this equation and its Hermitian conjugate, it is
simple to show that $\t{U}^\dagger \t{J} \t{U}$ is a conserved quantity;
in fact, it is the Hamiltonian of the system.  If it is chosen initially
to be the identity, then \t{U} everywhere satisfies the equation
\begin{equation}
\t{U}^\dagger \t{J} \t{U} = \t{J}
\label{eq:symplectic1}
\end{equation}
We can express \t{U} in terms of $M \times M$ blocks,
\begin{equation}
\t{U} = \left( \begin{matrix} \t{U}_{11} & \t{U}_{12} \\
\t{U}_{21} & \t{U}_{22} \end{matrix} \right)
\label{eq:u_blocks}
\end{equation}
in which the upper blocks represent displacements, the lower blocks
represent their conjugate momenta, and the left and right halves
represent different sets of initial conditions.  In terms of these
blocks, the symplectic property is equivalent to
\begin{eqnarray}
\t{U}_{11}^\dagger \t{U}_{21} = \t{U}_{21}^\dagger \t{U}_{11} \nonumber \\
\t{U}_{12}^\dagger \t{U}_{22} = \t{U}_{22}^\dagger \t{U}_{12} \nonumber \\
\t{U}_{11}^\dagger \t{U}_{22} - \t{U}_{21}^\dagger \t{U}_{12} = \t{I}
\label{eq:symplectic2}
\end{eqnarray}
Similarly, one can show that if 
\begin{equation}
\t{U} \t{J} \t{U}^\dagger = \t{J}
\label{eq:symplectic3}
\end{equation}
initially, then it retains this property, and from this we obtain the
additional independent relations,
\begin{eqnarray}
\t{U}_{11} \t{U}_{12}^\dagger = \t{U}_{12} \t{U}_{11}^\dagger \nonumber \\
\t{U}_{21} \t{U}_{22}^\dagger = \t{U}_{22} \t{U}_{21}^\dagger \nonumber \\
\t{U}_{11} \t{U}_{22}^\dagger - \t{U}_{21} \t{U}_{12}^\dagger = \t{I}.
\label{eq:symplectic4}
\end{eqnarray}
These symmetries play a central role in proving the theorems of Section
\ref{sec:fixed}.

\section{\label{sec:sing}Singular Surfaces}

The Euler-Lagrange equation has regular singular points at the resonant
surfaces where $q=m/n$, at the magnetic axis, and at a separatrix if one
exists.  In this section we discuss the asymptotic behavior of the
solutions in the neighborhood of each of these singular points.

The form of the Euler-Lagrange equation, Eqs. (\ref{eq:define_L}) and
(\ref{eq:euler2}), shows that the coefficient matrix \t{L} becomes
singular wherever the determinant of \t{F} vanishes.  Using the forms of
these matrices given in Eqs. (\ref{eq:simplify_f}) and
(\ref{eq:simplify_k}) of Appendix \ref{sec:coef}, we can write
\[
\t{F} = \t{Q} \bar{\t{F}} \t{Q}, \qquad 
\t{K} = \t{Q} \bar{\t{K}}, \qquad
\t{G} = \bar{\t{G}},
\]
where $\bar{\t{F}}$, $\bar{\t{G}}$, and $\bar{\t{K}}$ are nonsingular at
the resonant surfaces.  Since \t{Q} becomes singular at the resonant
surfaces where $q=m/n$, these are singular points.  For an axisymmetric
torus, as treated here, the toroidal mode number $n$ is a good quantum
number, {\it i.e.} a fixed integer for each perturbation treated, and
therefore the resonant surfaces are discrete.  In the presence of a
magnetic separatrix, where $q \to \infty$ logarithmically, there is a
point of accumulation of the resonant surfaces.  While we have not been
able to prove it analytically, we find numerically that $\bar{\t{F}}$
vanishes linearly with $\psi$ and $\bar{\t{G}}$ varies as $\psi^{-1}$ as
$\psi \to 0$; this corresponds to the behavior of Newcomb's equation for
the cylinder, making the axis a singular point.  In addition, we find
numerically that $\det \bar{\t{F}}$ vanishes at a separatrix, making it
another singular point.  In certain degenerate cases, {\it e.g.} if $q =
m/n$ at the axis or at an extremum where $q'$ vanishes, two or more
singular points can merge to form a singularity of higher rank.  We
treat only the relatively simple cases for which this does not occur.

In analyzing the asymptotic behavior of the solution in the neighborhood
of a resonant surface, we must take account of the full matrix nature of
the equations and the coupling between resonant and nonresonant
components.  To accomplish this, we use a matrix form of the Frobenius
analysis due to Turrittin.\cite{hlt55} Before treating the full case, we
illustrate the procedure for Newcomb's cylindrical case, for which the
Euler-Lagrange equation is
\begin{equation}
(f \xi')' - g \xi = 0,
\label{eq:newcomb1}
\end{equation}
where primes denote derivatives with respect to $r$.  To express this in
matrix form, let
\[
\v{u} = \left( \begin{matrix} \xi \\ f \xi' \end{matrix} \right), \qquad
\t{L} = \left( \begin{matrix} 0 & 1/f \\ g & 0 \end{matrix} \right).
\]
Then Eq. (\ref{eq:newcomb1}) becomes $\v{u}' = \t{L} \v{u}.$ Let $r_s$
be the position of the singular surface, let $z \equiv r - r_s$ for $r >
r_s$, and reinterpret the primes to denote derivatives with respect to
$z$.  The functions $f(r)$ and $g(r)$ can be Taylor expanded about the
singular surface as $f = f_0 z^2 + \cdots$,$g = g_0 + \cdots.$ Then to
lowest order,
\[
\t{L} = \left( \begin{matrix} 0 & 1/f_0 z^2 \\ g_0 & 0
	\end{matrix} \right) + \cdots. 
\]
We now define a new dependent variable vector \v{v} by introducing the
shearing transformation,
\begin{equation}
\v{u} = \t{R} \v{v}, \qquad 
\t{R} \equiv \left( \begin{matrix} z^{-1/2} & 0 \\ 0 & z^{1/2}
	\end{matrix} \right)
\label{eq:shear_cyl}
\end{equation}
Then \v{v} satisfies the equation $z \v{v}' = \t{M} \v{v}$ in terms of
the transformed matrix
\[
\t{M} = z \t{R}^{-1} (\t{L} \t{R} - \t{R}')
= \left( \begin{matrix} 1/2 & 1/f_0 \\ g_0 & -1/2 \end{matrix} \right) +
\cdots.
\]
If we now seek power-like solutions of the form $\v{v} = z^\alpha
\v{v}^{(0)} + \cdots$, then to lowest order we obtain the matrix
eigenvalue equation
\begin{equation}
(\t{M} - \alpha \t{I}) \v{v} = 0
\label{eq:lowest_cyl}
\end{equation}
with eigenvalues satisfying the characteristic equation,
\begin{eqnarray}
&&\det (\t{M} - \alpha \t{I}) = 0, \qquad
\alpha_\pm = \pm \sqrt{-D_I}, \nonumber \\
&&D_I = - g_0/f_0 - 1/4,
\label{eq:cyl_eigvals}
\end{eqnarray}
and eigenvectors
\[
\v{v}_\pm = \left( \begin{matrix} 1 \\ (\alpha_\pm - 1/2) f_0
\end{matrix} \right). 
\]
Equation (\ref{eq:lowest_cyl}) thus plays the role of the indicial
equation in the standard Frobenius analysis and also determines the form
of the solution vectors.  Transforming these solutions back to the
original vector of dependent variables \v{u} by means of
Eq. (\ref{eq:shear_cyl}), we find that the general solution can be
expressed as a linear combination of large and small solutions,
\begin{eqnarray}
\v{u}  =&& c_- z^{\alpha_-} 
	\left( \begin{matrix} z^{-1/2} \\ 
	z^{1/2} (\alpha_- - 1/2) f_0 \end{matrix} \right) \nonumber \\
&&+ c_+ z^{\alpha_+} 
	\left( \begin{matrix} z^{-1/2} \\ 
	z^{1/2} (\alpha_+ - 1/2) f_0 \end{matrix} \right)
\end{eqnarray}

The shearing transformation introduced in Eq. (\ref{eq:shear_cyl}) is
required in order to properly balance the components of the new
dependent variable vector \v{v}, giving both components the same
power-like behavior.  If we applied the same shearing transformation at
an ordinary point of the equation, or to a nonresonant component of the
solution at a resonant surface, then since the components of \v{u} are
already in balance, both approaching a constant, the shearing
transformation would unbalance them.  Thus, in treating the toroidal
case, we must choose a shearing transformation which appropriately
balances the resonant components without unbalancing the nonresonant
components.  This introduces half-integral powers into the coupling
between resonant and nonresonant components.

In the neighborhood of a resonant surface of a toroidal plasma, let
$q(\psi_R) = m/n$, $z \equiv \psi - \psi_R $ Then the dominant terms in
\t{L}, Eq. (\ref{eq:define_L}), vary as
\begin{eqnarray}
&&\t{F}^{-1} \sim z^{-2}, \nonumber \\
&&\t{F}^{-1} \t{K} \sim \t{K}^\dagger \t{F}^{-1} \sim z^{-1}, \nonumber \\
&&\t{G} \sim \t{K}^\dagger \t{F}^{-1} \t{K} \sim 1
\end{eqnarray}
To balance the equations, we introduce the shearing transformation
\begin{eqnarray}
\v{u} &&= \v{R} \v{v}, \qquad R_{m,m'} = z^{r_m} \delta_{m,m'},
	\nonumber \\
r_m &&= -1/2, \qquad r_{m+M} = 1/2 \qquad {\rm for\ } \ m = n q,
	\nonumber \\
r_m &&= r_{m+M} = 0 \qquad {\rm for\ } \ m \ne n q,
\label{eq:shear_tor}
\end{eqnarray}
where $M$ is the number of retained Fourier components.  Then the
equation is transformed to
\begin{equation}
z \v{v}' = \t{M} \v{v}.
\label{eq:transformed_eq_toroidal}
\end{equation}
The transformed matrix,
\[
\t{M} = z \t{R}^{-1} (\t{L} \t{R} - \t{R}' ),
\]
can be Taylor expanded about the resonant point as
\begin{equation}
\t{M} = \sum_{k=0}^\infty z^{k/2} \t{M}_k.
\label{eq:expand_M_tor}
\end{equation}
We now seek power-like solutions of the form 
\begin{equation}
\v{v} = z^\alpha \sum_{k=0}^\infty z^{k/2} \v{v}^{(k)}.
\label{eq:expand_v_tor}
\end{equation}
Introducing Eqs. (\ref{eq:expand_M_tor}) and (\ref{eq:expand_v_tor})
into Eq. (\ref{eq:transformed_eq_toroidal}), we obtain, as the
lowest-order terms, the matrix eigenvalue equation, $[\t{M}_0 - \alpha
\t{I}] \v{v}^{(0)} = 0$.  The eigenvalues $\alpha$ are the solutions to
the characteristic equation,
\begin{equation}
\det [\t{M}_0 - \alpha \t{I}] = 0.
\label{eq:det_M}
\end{equation}
Higher-order terms in the power series solutions can be obtained
iteratively from the higher-order equations,
\[
[\t{M}_0 - (\alpha + k/2) \t{I}] \v{v}^{(k)} 
	= - \sum_{l=1}^k \t{M}_l \v{v}^{(k-l)}
\]
Since this is a regular singular point, the powers series solutions are
convergent.

The spectrum of Eq. (\ref{eq:det_M}) is simple.  Since
$\t{M}_0$ is a $2M \times 2M$ matrix, there are $2M$ eigenvalues and
eigenvectors.  Of these, two represent the resonant solutions
corresponding to Eq. (\ref{eq:cyl_eigvals}) for the cylinder.  The other
$2M-2$ eigenvalues are zero, corresponding to the nonresonant solutions.

Determination of the resonant eigenvalues can be reduced to the solution
of a $2 \times 2$ characteristic equation.  We take two basis vectors,
each containing 1 for one of its resonant components and all other
components zero.  Multiplication of these vectors by $\t{M}_0$
annihilates all contributions from the null space.  The resonant
components of the resulting vectors form a $2 \times 2$ matrix whose
eigenvalues are
\begin{equation}
\alpha = \pm \sqrt{-D_I},
\label{eq:define_alpha}
\end{equation}
where $D_I$ is the well-known expression for the Mercier criterion in
axisymmetric toroidal geometry,\cite{jmg62,cm61,ahg75}
\begin{widetext}
\begin{eqnarray}
D_I =&& - \frac{1}{4} 
+ \frac{2 \pi f P' V'}{q' \chi'^3} 
	\left< \frac{1}{|\nabla \psi|^2} \right>
	\left[ 1 - \frac{2 \pi f P' V'}{q' \chi'^3}
	\left< \frac{1}{|\nabla \psi|^2} \right> \right] \nonumber \\
&&+ \left< \frac{B^2}{|\nabla \psi|^2} \right> 
	\frac{P' V'^2}{q'^2 \chi'^4}
	\left\{P' \left[ \left< \frac{1}{B^2} \right>
	+ \left( \frac{2 \pi f}{\chi'} \right)^2 
	\left< \frac{1}{B^2 |\nabla \psi|^2} \right> \right]
	+ \frac{\chi''}{\chi'} - \frac{V''}{V'} \right\}
\label{eq:di}
\end{eqnarray}
\end{widetext}
where $V$ is the volume inclosed within a flux surface.  While we have
not been able to prove analytically that the solutions to
Eq. (\ref{eq:det_M}) are in fact given by Eqs. (\ref{eq:define_alpha})
and (\ref{eq:di}), we find numerically that they converge to these
values as the number of retained Fourier components increases.
Agreement between these two ways of computing $D_I$ can thus serve as a
test for adequate Fourier convergence at a resonant surface.  The
Mercier criterion, that $D_I \le 0$ is a necessary condition for
stability, follows from that fact that if $D_I > 0$, then $\alpha$ is
imaginary, the solution vector \v{v} in Eq. (\ref{eq:expand_v_tor}) is
oscillatory, and the generalization of Newcomb's stability criterion,
presented in the next section, is repeatedly violated.

It would appear from Eqs. (\ref{eq:expand_M_tor}),
(\ref{eq:expand_v_tor}), and (\ref{eq:det_M}) that the higher-order
terms in the power-series solutions contain half-integral powers of $z$.
This is deceptive, because the shearing transformation in
Eq. (\ref{eq:shear_tor}) restores the half powers to integral powers for
the original variables \t{u}.  The real significance of the half powers
is that determination of higher order terms alternates between the
resonant and nonresonant components.  For example, the zeroth-order
resonant eigenvectors contain only resonant components, the first-order
terms are all nonresonant, the second-order terms are resonant, etc.
For the nonresonant eigenvectors, this order is reversed.  In principle,
the power series solutions can be determined to arbitrarily high order;
in practice, this is limited by the difficulty of evaluating the higher
order terms in the Taylor expansion of \t{M},
Eq. (\ref{eq:expand_M_tor}).

Once all of the independent eigenvectors have been determined to some
order, they can be assembled as the columns of a matrix which we denote
$\t{V}(\psi)$.  We choose a particular order for these vectors, as close as
possible to the identity.  The subspace of nonresonant eigenvectors,
corresponding to the eigenvalues $\alpha=0$, is spanned by a basis in
which each basis vector contains only one nonresonant component to
lowest order.  We arrange these vectors so that the nonvanishing
component appears on the diagonal and normalize them so that the
diagonal element is 1.  We place the large resonant eigenvector in the
resonant column of the left half of \t{V} and the small resonant
eigenvector in the resonant column of the left half of \t{V}.  With a
proper choice of normalization of the resonant solutions, \t{V} can be
made symplectic.

In principle, the convergent power series could be evaluated to all
orders, and $\t{V}$ would then be an exact fundamental solution matrix.
Any particular solution vector $\v{u}(\psi)$ of the exact Euler-Lagrange
equation, such as one obtained by integrating the equations numerically,
could then be expressed as a linear combination of these independent
solutions, $\v{u}(\psi) = \t{V}(\psi)\v{c}$, where the constant vector
$\v{c}$ consists of the expansion coeffients in a linear combination.
In practice, it is often more convenient to truncate the power series
after a few terms, in which case $\t{V}(\psi)$ is not an exact
fundamental solution matrix.  We can still express an exact solution as
a linear combination of the truncated power series solutions,
\begin{equation}
\v{u}(\psi) = \t{V}(\psi) \v{c}_A(\psi).
\label{eq:define_ca}
\end{equation}
Then the vector $\v{c}_A(\psi)$ of expansion coefficients is no longer
constant, but asymptotically approaches a limit as the resonant surface
is approached, as long as the number of terms in the power series is
sufficiently large.  We therefore denote the components of this vector
asymptotic coefficients.

As $\psi \to \psi_R$, $\v{c}_A$ can be obtained by solving
Eq. (\ref{eq:define_ca}).  The matrix $\t{V}(\psi)$ is used in the
following section in deriving the proofs of the basic stability
theorems.  The components of $\v{c}_A(\psi)$ are also involved in the
generalization of the quantity $\Delta'$ which appears in the study of
singular modes.  This will be the subject of a later paper.

The two resonant eigenvectors corresponding to the eigenvalues in
Eq. (\ref{eq:define_alpha}) have powers differing by $2 \sqrt{-D_I}$.
If accurate determination of the asymptotic coefficient of the small
resonant solution is necessary, as in the treatment of singular modes,
then enough terms in the series solution of the large solution must be
retained so that its remaining error is smaller than the zeroth-order
term in the small solution.  If $|D_I|$ is large, this can make the
procedure numerically impractical.  This typically occurs in regions of
very low shear $q'$, for example when $q$ has a turning point, or if
$\beta$, the ratio of plasma pressure to magnetic energy density, is
large.

The axis is a singular point of the Euler-Lagrange equation, just at it
is for Newcomb's cylindrical case.  For the latter, the solutions for $r
\xi_r$ vary as $r^{\pm m}$ in the neighborhood of the axis.  Boundary
conditions at the axis must exclude the $r^{-m}$ solutions and retain
the $r^{m}$ solutions.  Similarly, one would expect that, in the
toroidal case, solutions for $\xi \cdot \nabla \psi$ should vary as
$\psi^{\pm \mu/2}$, with $\mu$ an integer.  Determination of the
asymptotic behavior at the axis should be reducible to an analysis
similar to that presented above for the resonant surfaces, based on a
matrix eigenvalue equation of the form
\begin{equation}
\lim_{\psi \to 0} [ \psi \t{L} - \alpha \t{I} ] \t{u} = 0.
\label{eq:axis}
\end{equation}
There are several complications which make this difficult.  The complex
form of $\t{F}$ given in Eq. (\ref{eq:simplify_f}), especially coupling
to the $m=0$ term, make it difficult to extract the asymptotic behavior
of \t{L} analytically.  Since the eigenvalues differ by integers, there
are further complications discussed by Turrittin.

Equation (\ref{eq:axis}) can be treated numerically as the solution to a
matrix eigenvalue problem.  Doing so, we find that the eigenvalues do
indeed approach $\mu/2$, as expected.  The eigenvectors obtained by this
means are used to initialize the integration near the axis.  In general,
a pure eigenvector, corresponding to pure $\psi^{\mu/2}$ behavior,
contains a mixture of all Fourier components $m$, due to toroidal ($\pm
1$) and elliptic ($\pm 2$) coupling in the neighborhood of the axis.

If a separatrix bounds the region of confined plasma, it has two effects
on the distribution of singular surfaces.  First, it causes $q \to
\infty$, introducing an accumulation point of singular surfaces due to
the $\t{Q}$ factors in \t{F}, Eq. (\ref{eq:simplify_f}).  Second, we
have found numerically that the remaining factors of \t{F} have a
vanishing determinant at the axis.  If we treat a truncated Fourier
series with some maximum value of $m$, then this additional singularity
can be regarded to play a role similar to that of the singular point at
the axis.  A full understanding of this behavior remains to be achieved.

At each resonant point, one eigenvalue of \t{F} has a quadratic zero; it
does not change sign.  The only other singularities of \t{F} are at the
axis and the separatrix.  Thus, throughout the region of closed field
lines between the axis and the separatrix, all eigenvalues of \t{F} are
non-negative.  This property is used in the next section to generalize
the proofs of the stability theorems.

\section{\label{sec:fixed}Fixed-Boundary Modes}

Our goal in this paper is to determine whether there exist test
functions $\Xi_\psi(\psi)$ which make $\delta W$ negative while
satisfying the boundary conditions.  In our analysis so far, we have
derived the Euler-Lagrange equation which makes $\delta W$ stationary
with respect to variations of $\Xi_\psi$, and we have studied its
symmetry properties and its singular points.  Now we must understand the
relationship between the solutions to this equation and the existence of
test functions which make $\delta W$ negative.

Here is an outline of the proofs in this section.  We first consider the
case of a plasma bounded by flux surfaces $\psi_1$ and $\psi_2$ which
are both ordinary points of the Euler-Lagrange equation, with no
resonant points between them.  We study a 2-point boundary value problem
on the interval $[\psi_1,\psi_2]$.  We use this to find the slope of the
Euler-Lagrange solution passing through any point in the space of
solutions.  This motivates the definition of the plasma response matrix
$\t{W}_P(\psi)$ and the critical determinant $D_C(\psi) \equiv \det
\t{W}_P(\psi)$.  If $D_C$ has no poles in $[\psi_1,\psi_2]$, we use the
results of the 2-point boundary value problem to construct a Hilbert
invariant integral and prove that it depends only on the boundary
conditions and that it constitutes a lower bound on $\delta W$.  The
lower bound is achieved if and only if $\Xi(\psi)$ is an Euler-Lagrange
solution satisfying the boundary conditions.  If $\psi_1$ and $\psi_2$
are perfect conductors, this lower bound is shown to vanish, proving
that the absence of poles in $D_C$ is a sufficient condition for
stability.  If $D_C$ has a pole in $[\psi_1,\psi_2]$, we prove by
construction that there exists a test function $\Xi(\psi)$ which makes
$\delta W < 0$, and thus that the absence of poles in $D_C$ is a
necessary condition for stability.  Next we generalize the proofs to the
case where the plasma is bounded by the axis, $\psi=0$; and by a
resonant surface where $q(\psi_R)=m/n$.  Finally, we consider the case
of one or more resonant points $\psi_R \in [\psi_1,\psi_2]$.  We show
that if $D_C$ has no pole in $[\psi_1,\psi_R)$ or in $(\psi_R,\psi_2]$,
then the lower bound on $\delta W$ is achieved if and only if
$\Xi(\psi)$ is an Euler-Lagrange solution in each nonresonant
subinterval, contains no large resonant solution at $\psi_R$, and has
nonresonant contributions which are continuous with continuous
derivatives.  Generalizations of the necessary and sufficient conditions
then follow easily.  The case where the plasma is bounded by a vacuum
region is considered in the next section.

As a preliminary, we treat a 2-point boundary value problem.  Consider
an interval $\psi_1 \le \psi \le \psi_2$ containing no singular points
where $\det \t{F}=0$.  The solutions to the Euler-Lagrange equation in
this interval constitute a $2M$-dimensional complex vector space.  A
unique solution can be selected by imposing $2M$ boundary conditions.
These could be imposed entirely at $\psi_1$, entirely at $\psi_2$, or
any combination thereof.  We impose half of the conditions at $\psi_1$
and half at $\psi_2$, with all conditions imposed on $\Xi$, the upper
half of $\v{u}$ as defined in Eq. (\ref{eq:define_u}).  We define a
fundamental matrix of solutions satisfying
\begin{equation}
\t{U}'(\psi) = \t{L}(\psi) \t{U}(\psi), \qquad \t{U}(\psi_1) = \t{I}.
\label{eq:define_U}
\end{equation}
Any particular solution $\v{u}$ can be expressed as a linear combination
of the columns of $\t{U}$,
\begin{equation}
\v{u}(\psi) = \t{U}(\psi) \v{c},
\label{eq:define_c}
\end{equation}
where \v{c} is a constant vector of expansion coefficients.  Determining
\v{u} is equivalent to determining \v{c}.  The boundary conditions are
given by $\Xi(\psi_1) = \v{u}_1(\psi_1) = \Xi_1$, $\Xi(\psi_2) =
\v{u}_1(\psi_2) = \Xi_2$, where $\v{u}_1$ refers to the upper half of
\v{u} and the $\Xi$'s are constant $M$-vectors.  Using the decomposition
of $\t{U}$ given in Eq. (\ref{eq:u_blocks}), we can express the boundary
conditions in terms of \v{c} as $\v{c}_1 = \Xi_1$, $\t{U}_{11}(\psi_2)
\v{c}_1 + \t{U}_{12}(\psi_2) \v{c}_2 = \Xi_2$, where $\v{c}_1$ and
$\v{c}_2$ are the upper and lower halves of \v{c}.  Since the boundary
conditions at $\psi_1$ determine $\v{c}_1$, the boundary conditions at
$\psi_2$ must be satified by solving for $\v{c}_2$,
\begin{equation}
\v{c}_2 = \t{U}_{12}^{-1}(\psi_2) [\Xi_2 - \t{U}_{11}(\psi_2) \Xi_1]. 
\label{eq:define_c2_1}
\end{equation}
From this it is clear that the existence of a solution to the boundary
value problem depends on whether $\t{U}_{12}(\psi_2)$ is invertible,
{\it i.e.} has a nonvanishing determinant.  If it does not, then the
solution exists only if the quantity in brackets is orthogonal to the
null space of $\t{U}_{12}^\dagger$, in which case the solutions are not
unique, but there is a family of solutions passing through the same
point.

Following Newcomb, we wish to define a Hilbert invariant integral and
determine when it exists and constitutes a lower bound on $\delta W$.
For the cylindrical case, Newcomb defines the Hilbert invariant integral
as a contour integral in a 2-dimensional real vector space whose axes
are $\xi$ and $r$; we define our generalization as a contour integral in
a space which is a cartesian product of one real dimension for $\psi$
and an $M$-dimensional complex vector space of $\Xi$'s.  Newcomb covers
his 2-space with Euler-Lagrange solutions which all satisfy a given left
boundary condition $\xi(r_1) = \xi_1$; we cover our space with solutions
which all satisfy a given left boundary condition $\Xi(\psi_1) = \Xi_1$.
Newcomb defines his contour integral in terms of the slope of the
Euler-Lagrange solution passing through a given point in $\xi$-$r$
space.  In our space, this involves finding the Euler-Lagrange solution
which passes through $\Xi_1$ at $\psi_1$ and through $\Xi$ at $\psi$,
the solution to our the 2-point boundary value problem.  Its slope is
given by
\begin{eqnarray}
\Xi^*(\psi,\Xi_1,\Xi) \equiv \t{F}^{-1} \left\{ 
	[\t{U}_{21}(\psi) \Xi_1 - \t{K}(\psi)\Xi] \right. \nonumber \\
+\left. \t{U}_{22}(\psi) \t{U}_{12}^{-1}(\psi) 
	[\Xi - \t{U}_{11}(\psi) \Xi_1] \right\}.
\label{eq:define_xi*}
\end{eqnarray}

We define the plasma response matrix $\t{W}_P$ and the critical
determinant $D_C$ as
\begin{equation}
\t{W}_P(\psi) \equiv \t{U}_{22}(\psi) \t{U}_{12}^{-1}(\psi), \quad
D_C(\psi) \equiv \det \t{W}_P(\psi).
\label{eq:define_dc}
\end{equation}
The plasma response matrix occurs in the expression for $\Xi^*$ in
Eq. (\ref{eq:define_xi*}).  Since $\det \t{F}$ never vanishes in the
interval $[\psi_1,\psi_2]$, the existence of $\Xi^*$ throughout the
interval depends only on whether $\t{W}_P$ is well-behaved, {\it i.e.}
whether its determinant $D_C$ has poles in the interval.  From the
symplectic symmetry properties in Eqs. (\ref{eq:u_blocks}) and
(\ref{eq:symplectic2}), we find that $\t{W}_P$ is self-adjoint and
therefore $D_C$ is real.  We shall see that poles in $D_C$ constitute
the appropriate generalization of Newcomb's criterion involving zeroes
of his real scalar Euler-Lagrange solution $\xi(r)$.  We can also
express our generalized criterion in a manner more similar to Newcomb's
by noting that the smallest eigenvalue of $\t{W}_P^{-1}$ has a zero
wherever $D_C$ has a pole.  In fact, this is a more useful numerical
diagnostic.

If $D_C$ has no poles in the interval $[\psi_1,\psi_2]$, we define our
Hilbert invariant integral as a contour integral in $\Xi-\psi$ space,
\begin{eqnarray}
W^\dagger(C) \equiv&& \int_C  \Bigl\{
	[\Xi^{*\dagger} (\t{F} \Xi^* + \t{K} \Xi)
	+ \Xi^\dagger (\t{K}^\dagger \Xi^\star + \t{G} \Xi) ] d\psi 
	\nonumber \\
&&+ (d\Xi^\dagger - \Xi^{*\dagger} d\psi) 
	(\t{F} \Xi^* + \t{K} \Xi) \nonumber \\
&&+ (\Xi^{*\dagger} \t{F} + \Xi^\dagger \t{K}^\dagger) 
	(d\Xi - \Xi^* d\psi) \Bigr\} \nonumber \\
=&& \int_C \left[ w_\psi d\psi 
	+ d\Xi^\dagger \v{w}_{\Xi^\dagger} 
	+ \v{w}_\Xi d\Xi \right].
\label{eq:define_w_dagger}
\end{eqnarray}
where $C$ is any contour connecting the boundary conditions in our
$(M+1)$-dimensional $\Xi-\psi$ space and
\begin{equation}
w_\psi \equiv \Xi^\dagger \t{G} \Xi - \Xi^{* \dagger} \t{F} \Xi^*, \quad
\v{w}_{\Xi^\dagger} = \v{w}_\Xi^\dagger = \t{F} \Xi^* + \t{K} \Xi.
\label{eq:define_w_psi}
\end{equation}

To prove that $W^\dagger(C)$ is independent of the contour and depends
only on the endpoints, we must prove that the integrand is curl-free in
this multidimension space.  This follows from the conditions
\[
\frac{\partial w_\psi}{\partial \Xi^\dagger} 
= \frac{\partial w_{\Xi^\dagger}}{\partial \psi}, \quad
\frac{\partial w_{\Xi^\dagger}}{\partial \Xi}
= \frac{\partial w_\Xi}{\partial \Xi^\dagger},
\]
which follow easily from Eq. (\ref{eq:define_w_psi}).  To prove that
$W^\dagger(C)$ constitutes a lower bound for $\delta W$, we note that
\[
\delta W - W^\dagger(C) = \int_C d\psi (\Xi' - \Xi^*)^\dagger \t{F}
	(\Xi' - \Xi^*) \ge 0,
\]
because \t{F} is non-negative, as discussed at the end of the previous
section.  Equality holds if and only if $\Xi$ is an Euler-Lagrange
solution.

Having shown that $\delta W$ is minimized by the unique Euler-Lagrange
solution satisfying the boundary condition if $D_C$ has no poles, we now
find the value of that minimum.  Starting with Eq. (\ref{eq:dw4}), we
first symmetrize the expression to obtain
\begin{eqnarray}
\delta W =&& \frac{1}{2} \int_{\psi_1}^{\psi_2} d\psi \big[ 
	\Xi'^\dagger (\t{F} \Xi' + \t{K} \Xi) 
	+ \Xi^\dagger (\t{K}^\dagger \Xi' + \t{G} \Xi) \nonumber \\
&&+ ( \Xi'^\dagger \t{F} + \Xi^\dagger \t{K}^\dagger) \Xi'
	+ ( \Xi'^\dagger \t{K} + \Xi^\dagger \t{G}) \Xi \big].
\label{eq:dw_sym}
\end{eqnarray}
Next we use the Euler-Lagrange equation, Eq. (\ref{eq:euler1}), and its
Hermitian conjugate, to eliminate terms in Eq. (\ref{eq:dw_sym})
containing $\t{G}$ and obtain
\begin{eqnarray}
\delta W
=&& \frac{1}{2} \int_{\psi_1}^{\psi_2} d\psi \left[ 
	\Xi'^\dagger (\t{F} \Xi' + \t{K} \Xi) 
	+ \Xi^\dagger (\t{F} \Xi' + \t{K} \Xi)' \right. \nonumber \\
&&\left. + ( \Xi'^\dagger \t{F} + \Xi^\dagger \t{K}^\dagger) \Xi'
	+ ( \Xi'^\dagger \t{F} + \Xi^\dagger \t{K}^\dagger)' 
	\Xi \right] \nonumber \\
=&& \frac{1}{2} \int_{\psi_1}^{\psi_2} d\psi \left[ 
	\Xi^\dagger (\t{F} \Xi' + \t{K} \Xi) 
	+ (\Xi'^\dagger \t{F} + \Xi^\dagger \t{K}^\dagger)
	\Xi \right]' \nonumber \\
=&& \left. \frac{1}{2} \left[ 
	\Xi^\dagger (\t{F} \Xi' + \t{K} \Xi)
	+ (\Xi'^\dagger \t{F} + \Xi^\dagger \t{K}^\dagger) \Xi
	 \right] \right|_{\psi_1}^{\psi_2} \nonumber \\
=&& \left. \frac{1}{2} \left( \t{u}_1^\dagger \t{u}_2 
	+ \t{u}_2^\dagger \t{u}_1 \right) \right|_{\psi_1}^{\psi_2}
\label{eq:dw_boundary}
\end{eqnarray}

	The unique Euler-Lagrange solution satisfying the boundary
conditions is the solution to the 2-point boundary value problem above.
If the plasma response matrix is nonsingular throughout the interval,
then $\v{u}_2(\psi)$ is bounded everywhere.  If $\v{u}_1=\Xi$ vanishes
at the boundaries, as it must at perfectly conducting walls, then this
minimizing $\delta W$ vanishes, proving that there are no test functions
which make $\delta W$ negative.  This proves the sufficient condition
for stability, that in the absence of poles in $D_C$, the system is
stable.  So far the proof applies only to the case where $\psi_1$ and
$\psi_2$ are conducting walls.  This is further generalized below.

Next we prove, for the same case, that the absence of poles in $D_C$ is
a necessary condition for stability, by showing that if $D_C$ has a
pole, we can construct a test function which makes $\delta W$ negative.
If $\v{c}_1 = \Xi_1 = 0$, then $\v{u}_1(\psi) = \t{U}_{21}(\psi)
\v{c}_2$, $\v{u}_2(\psi) = \t{U}_{22}(\psi) \v{c}_2 = \t{U}_{22}(\psi)
\t{U}_{21}^{-1}(\psi) \v{u}_1(\psi) = \t{W}_P(\psi) \v{u}_1(\psi)$, and
$\v{u}_1(\psi) = \t{W}_P(\psi)^{-1} \v{u}_2(\psi)$.  If $D_C(\psi)$ has
a pole at $\psi=\psi_P$, then $\t{W}_P^{-1}(\psi_P)$ has a vanishing
determinant and hence a nontrivial null space.  If we choose
$\v{u}_2(\psi_P)$ to be in that null space, the corresponding
$\v{u}_1(\psi_P)$ vanishes.  We may therefore construct a test function
which is nonzero in $[\psi_1,\psi_P]$, for which $\v{u}_1(\psi)$
vanishes at both $\psi_1$ and $\psi_P$, and therefore, by
Eq. (\ref{eq:dw_boundary}), the contribution of this interval to $\delta
W$ vanishes.  If we make the test function vanish identically on
$[\psi_P, \psi_2]$, then the contribution from that interval also
vanishes.  This test function is an Euler-Lagrange solution on each of
the subintervals, but not on the whole interval $[\psi_1,\psi_2]$.  In
particular, it is not an Euler-Lagrange solution on any subinterval
containing $\psi_P$.  We may choose a small subinterval containing
$\psi_P$ and connect the points on the original test function on either
side of $\psi_P$ with an Euler-Lagrange solution between those points.
This must give a lower contribution to $\delta W$ than the original test
function.  Since $\delta W$ vanishes for the original test function, the
new trial function makes $\delta W$ negative.  This completes the proof.

To generalize the proofs to the case where $\psi_1 = 0$, the magnetic
axis, we replace the initial conditions on \t{U}.  Rather than
initialize to the identity, as in Eq. (\ref{eq:define_U}), we construct
an initial \t{U} from the asymptotic solutions in the neighborhood of
the axis, the solutions to Eq. (\ref{eq:axis}).  Let the left half of
\t{U} contain the solutions which are asymptotically large at the axis
and the right half contain those which are small.  Then the constant
vector \v{c} defined in Eq. (\ref{eq:define_c}) constitutes the
expansion coefficients of these asymptotic solutions, the top half
$\v{c}_1$ the coefficients of the irregular solutions and the bottom
half $\v{c}_2$ the coefficients of the regular solutions.  The boundary
conditions at the axis require that $\v{c}_1 = 0$.  The rest of the
proofs follow unchanged.

To understand the behavior as $\psi_2$ approaches a resonant surface
$\psi_R$, we first consider the behavior of the plasma response matrix
$\t{W}_P$.  For the cylindrical case, with $M=1$, $\t{W}_P$ is a scalar,
the ratio of the second to the first component of the 2-vector \v{u}
which is small at the axis.  From the shearing transformation in
Eq. (\ref{eq:shear_cyl}) or (\ref{eq:shear_tor}), we see that this
vanishes, $\t{W}_P \sim z=\psi_R-\psi$ as $z \to 0$.  From
Eq. (\ref{eq:dw_boundary}), with $\Xi_1=0$ because $\psi_1$ is either a
conducting wall or the axis, we find
\begin{equation}
\delta W 
= \left. \frac{1}{2} \left( 
	\v{u}_1^\dagger \v{u}_2 + \v{u}_2^\dagger \v{u}_1
	 \right) \right|_{\psi_1}^{\psi_2}
= \left. \Xi^\dagger \t{W}_P \Xi \right|_{\psi_2}.
\label{eq:define_wp}
\end{equation}
Thus, if $\Xi$ were bounded at the resonant surface, $\delta W$ would
vanish.  To keep $\delta W$ bounded, it is overly restrictive to require
that $\Xi$ bounded; it would be sufficient to let $\Xi \sim z^{-1/2}$.
Since the minimizing test function is an Euler-Lagrange solution, we may
take $\Xi$ to be a linear combination of the large and small resonant
solutions, varying as $z^{-1/2 \pm \sqrt{-D_I}}$.  Since the large
solution diverges more rapidly than $z^{-1/2}$, it would contribute a
positive-definite infinite term to $\delta W$, precluding the
determination of a minimum, and must therefore be excluded.  Then only
the small resonant solution can be allowed, and its contribution to
$\delta W$ vanishes.  For the general case $M > 1$, $\t{W}_P$ has one
vanishing eigenvalue and $M-1$ eigenvalues approaching constants.  The
contribution from the large resonant solution is required to vanish, as
a boundary condition at the resonant surface.  In the cylindrical case,
the only solution which is small at both the axis and the resonant
surfaces is the trivial solution which is identically zero throughout
the first interval.  In the general case, there may be an arbitrary
linear combination of nonresonant terms, spanning a space of dimension
$M-1$, which make finite contributions to $\delta W$, according to Eq.
(\ref{eq:define_wp}).  The vector $\Xi$ may be bounded or may have an
unbounded contribution from the small resonant solution, and is
otherwise arbitrary.  Only the nonresonant terms make a finite
contribution to $\delta W$.

To treat the boundary conditions at a resonant surface $\psi_2 =
\psi_R$, we note that a particular solution can be expressed in terms of
two different bases, $\v{u}(\psi) = \t{U}(\psi) \v{c} = \t{V}(\psi)
\v{d}$.  Here, \t{U} is the fundamental solution matrix defined above,
initialized to the large and small solutions at the axis; \t{V} is the
fundamental solution matrix defined in Section III above
Eq. (\ref{eq:define_ca}), constructed from the full convergent power
series solutions about the resonant surface; \v{c} is a constant vector
whose upper and lower halves $\v{c}_1$ and $\v{c}_2$ contain the
expansion coefficients of solutions which are large and small at the
axis, respectively; and \v{d} is a constant vector whose upper half
$\v{d}_1$ contains the components of the nonresonant displacements and
the coefficient of the large resonant solution at the resonant surface,
and whose lower half $\v{d}_2$ contains the second components of the
nonresonant solutions and the coefficient of the small resonant solution
at the resonant surface.  The boundary condition at the axis is
$\v{c}_1=0$.  We wish to determine $\v{c}_2$ by imposing boundary
conditions at the resonant surface, specifying $\v{d}_1$, imposing an
arbitrary value for the nonresonant displacements and zero for the
coefficient of the nonresonant solution.  This can be done by letting
$\v{d} = \t{V}^{-1} \t{U}\v{c}$, letting $d_1 = \Xi_R$ with vanishing
resonant component, using the symplectic property,
Eq.(\ref{eq:symplectic1}), to write $\t{V}^{-1}=-\t{J} \t{V}^\dagger
\t{J}$, and inverting to obtain
\begin{equation}
\v{c}_2 = (\t{V}_{22}^\dagger
\t{U}_{12}-\t{V}_{12}^\dagger \t{U}_{22})^{-1} \Xi_R,
\label{eq:define_c2_2}
\end{equation}
If the power series in \t{V} are truncated to some order, then $\v{c}_2$
may be evaluated as the limit as $\psi \to \psi_R$.  Equation
(\ref{eq:define_c2_2}) may be regarded as a generalization of
Eq. (\ref{eq:define_c2_1}) to a resonant surface, with $\Xi_1=0$ and
$\t{U}_{12}$ replaced by the appropriate limiting matrix.  The
nonresonant case is recovered by replacing $\t{V}_{22}$ with $\t{I}$ and
$\t{V}_{12}$ with $\t{0}$.  In the cylindrical case, the only solution
is the trivial one, $\v{c}_2=\v{0}$.  In the toroidal case, the
nonresonant components of $\Xi$ remain arbitrary, and thus $\v{c}_2$
spans a space of dimension $M-1$.  Equation (\ref{eq:define_c2_2}) is a
general result for any value of $\Xi_R$, including a nonzero resonant
component.  With a vanishing resonant component, it can be further
simplified to
\begin{equation}
\v{c}_2 = \lim_{\psi \to \psi_R} \t{U}_{12}^{-1}(\psi) \Xi_R,
\label{eq:define_c2_3}
\end{equation}
the same form as for the nonresonant case.

If $[\psi_1,\psi_2]$ contains one or more resonant surfaces, the above
proof of the sufficient condition is no longer valid.  We can no longer
construct $\Xi^*$, Eq. (\ref{eq:define_xi*}); or $W^\dagger$,
Eq. (\ref{eq:define_w_dagger}); or the Euler-Lagrange solutions,
Eq. (\ref{eq:define_U}); on the whole interval because $\t{F}^{-1}$
diverges at each resonant point.  We consider first the case of one
resonant point $\psi_R \in [\psi_1,\psi_2]$ and assume that $\Xi_1=0$
because $\psi_1$ is either the axis or a conducting wall.  In order to
determine the minimizing test function, we divide the interval into two
subintervals $[\psi_1,\psi_R]$ and $[\psi_R,\psi_2]$.  The perturbation
$\Xi(\psi_R) = \Xi_R$ may be bounded or it may have an unbounded
contribution from the small resonant solution.  The nonresonant terms
must be continuous across the singular surface in order to avoid
positive infinite contributions to $\delta W$, but are otherwise
arbitrary.  We compute the minimizing contributions to $\delta W$ from
each subinterval as a function of $\Xi_R$.  We then perform a formal
minimization with respect to $\Xi_R$ and show that the minimizing
solution is the one for which the nonresonant solutions have continuous
derivatives as well as values.  Since the contribution from the small
resonant solution in the neighborhood of $\psi_R$ vanishes, it remains
arbitrary, to be determined by boundary conditions for $\psi > \psi_R$.

For simplicity, we first study the procedure for the case where $\psi_R$
is an arbitrary reference point, an ordinary point, and then consider
how it must be modified if $\psi_R$ is a resonant point.  We expect that
for the ordinary point, the results should show that the solution is a
single, continuous Euler-Lagrange solution, and this is indeed the case.

In the subinterval $[\psi_1,\psi_R]$, we initialize the fundamental
solution matrix $\t{U}(\psi)$ to the identity at $\psi_1$ if $\psi_1$ is
a conductor; or to the asymptotic solutions in the neighborhood of the
axis if $\psi_1=0$, with the large solutions in the left half and the
small solutions in the right half.  In either case, $\v{c}_1=0$,
$\v{u}_1(\psi) = \t{U}_{12}(\psi)\v{c}_2$, and $\v{u}_2(\psi) =
\t{U}_{22}(\psi) \v{c}_2$.  The boundary conditions at $\psi_R$ then
require that $\v{u}_1(\psi_R^-) = \Xi_R$, $\v{c}_2 =
\t{U}_{12}^{-1}(\psi_R^-) \Xi_R$, and $\v{u}_2(\psi_R^-) =
\t{U}_{22}(\psi_R^-)\t{U}_{12}^{-1}(\psi_R^-) \Xi_R$.  To simplify
notation, we define $\t{S} \equiv \t{U}(\psi_R^-)$; thus
$\v{u}_2(\psi_R^-) = \t{S}_{22} \t{S}_{12}^{-1} \Xi_R$.

In the subinterval $[\psi_R,\psi_2]$, we initialize the fundamental
solution matrix $\t{U}(\psi)$ to the identity at $\psi_R$.  The general
solution in this subinterval is $\v{u}_1 = \t{U}_{11} \v{d}_1 +
\t{U}_{12} \v{d}_2$, $\v{u}_2 = \t{U}_{21} \v{d}_1 + \t{U}_{22}
\v{d}_2$.  The boundary condition at $\psi_R^+$ determines that $\v{d}_2
= \Xi_R$, while that at $\psi_2$ determines that $\v{d}_2 =
\t{U}_{12}^{-1}(\psi_2)(\Xi_2 - \t{U}_{11} \Xi_R)$.  Again to simplify
notation, we define $\t{T} \equiv \t{U}(\psi_2)$.  Then
$\v{u}_2(\psi_R^+) = \t{T}_{12}^{-1} (\Xi_2 - \t{T}_{11} \Xi_R)$ and
$\v{u}_2(\psi_2) = \t{T}_{21}\Xi_R + \t{T}_{22} \t{T}_{12}^{-1}(\Xi_2 -
\t{T}_{11}\Xi_R)$.

Using the symplectic symmetry properties in Eqs. (\ref{eq:symplectic2})
and (\ref{eq:symplectic4}), we can express the total potential energy in
$[\psi_1,\psi_2]$ as
\begin{widetext}
\begin{eqnarray}
\delta W 
=&& \frac{1}{2} \big[ \v{u}_1^\dagger(\psi_R^-) \v{u}_2(\psi_R^-) 
	+ \v{u}_2^\dagger(\psi_R^-) \v{u}_1(\psi_R^-)
	- \v{u}_1^\dagger(\psi_R^+) \v{u}_2(\psi_R^+) 
	- \v{u}_2^\dagger(\psi_R^+) \v{u}_1(\psi_R^+) \nonumber \\
&& + \v{u}_1^\dagger(\psi_2) \v{u}_2(\psi_2) 
	+ \v{u}_2^\dagger(\psi_2) \v{u}_1(\psi_2) \big] \nonumber \\
=&& \Xi_R^\dagger \t{S}_{22} \t{S}_{12}^{-1} \Xi_R
	- \frac{1}{2} \Xi_R^\dagger \t{T}_{12}^{-1}
	(\Xi_2 - \t{T}_{11} \Xi_R)
	- \frac{1}{2} 
	(\Xi_2^\dagger - \Xi_R^\dagger \t{T}_{11}^\dagger )
	\t{T}_{12}^{-1\dagger} \Xi_R \nonumber \\
&& + \frac{1}{2} \Xi_2^\dagger [ \t{T}_{21} \Xi_R
	+ \t{T}_{22} \t{T}_{12}^{-1}
	(\Xi_2 - \t{T}_{11} \Xi_R) ]
	+ \frac{1}{2} [ \Xi_R^\dagger \t{T}_{21}^\dagger
	+ (\Xi_2^\dagger - \Xi_R^\dagger \t{T}_{11}^\dagger)
	\t{T}_{12}^{-1\dagger} \t{T}_{22}^\dagger ] \Xi_2 \nonumber \\
=&& \Xi_R^\dagger ( \t{S}_{22} \t{S}_{12}^{-1}
	+ \t{T}_{12}^{-1} \t{T}_{11} ) \Xi_R
	+ \Xi_2^\dagger \t{T}_{22} \t{T}_{12}^{-1} \Xi_2 \nonumber \\
&& - \frac{1}{2} \Xi_R^\dagger (
	\t{T}_{12}^{-1} - \t{T}_{21}^\dagger
	+ \t{T}_{11}^\dagger \t{T}_{12}^{-1\dagger} \t{T}_{22}^\dagger
	) \Xi_2
	- \frac{1}{2} \Xi_2^\dagger (
	\t{T}_{12}^{-1\dagger} - \t{T}_{21}
	+ \t{T}_{22} \t{T}_{12}^{-1} \t{T}_{11}
	) \Xi_R \nonumber \\
=&& \Xi_R^\dagger (\t{S}_{22} \t{S}_{12}^{-1}
	+ \t{T}_{12}^{-1} \t{T}_{11} ) \Xi_R
	- \Xi_R^\dagger \t{T}_{12}^{-1} \Xi_2
	- \Xi_2^\dagger \t{T}_{12}^{-1\dagger} \Xi_R
	+ \Xi_2^\dagger \t{T}_{22} \t{T}_{12}^{-1} \Xi_2.
\end{eqnarray}
\end{widetext}
Minimizing this with respect to $\Xi_R^\dagger$, we obtain
\begin{equation}
( \t{S}_{22} \t{S}_{12}^{-1}
	+ \t{T}_{12}^{-1} \t{T}_{11} ) \Xi_R
	= \t{T}_{12}^{-1} \Xi_2.
\label{eq:define_xir}
\end{equation}
This is just the condition that $\v{u}_2(\psi_R^-) = \v{u}_2(\psi_R^+)$,
{\it i.e.} that the whole solution be continuous at $\Xi_R$.  It is also
the condition that $\t{u}(\psi)$ be the same Euler-Lagrange solution on
both sides of the reference point.

Substituting Eq. (\ref{eq:define_xir}) into Eq. (\ref{eq:define_c2_3}),
we find
\begin{eqnarray}
\delta W =&& \Xi_2^\dagger \left[
	\t{T}_{22} \t{T}_{12}^{-1}  \right. \nonumber \\
&&\left. 
	- \t{T}_{12}^{-1\dagger}
	\left( \t{S}_{22} \t{S}_{12}^{-1}
	+ \t{T}_{12}^{-1} \t{T}_{11} \right)^{-1}
	\t{T}_{12}^{-1} \right] \Xi_2.
\end{eqnarray}
Again using the Eqs. (\ref{eq:symplectic2}) and (\ref{eq:symplectic2}),
we obtain finally
\begin{equation}
\delta W = \Xi_2^\dagger 
	\left( \t{T}_{21} \t{S}_{12} + \t{T}_{22} \t{S}_{22} \right)
	\left( \t{T}_{11} \t{S}_{12} + \t{T}_{12} \t{S}_{22} \right)^{-1}
	\Xi_2.
\label{eq:define_dw}
\end{equation}
The matrix product in Eq. (\ref{eq:define_dw}) is just the
plasma response matrix obtained by multiplying the fundamental solution
matrices to the left and right sides of $\psi_R$.

Now consider how this procedure may be adapted to the case where
$\psi_R$ is a resonant point.  We use Eq. (\ref{eq:define_c2_3}) for the
behavior to the left of the resonant point.  To the right, we initialize
$\t{U}(\psi)$ to $\t{V}(\psi)$, the matrix of power series solutions to
the right of the resonant point.  On both sides we take the resonant
contribution of $\Xi_R$ to vanish.  The small resonant component on
either side makes no contribution to $\delta W$.  Then the problem has
the same form as for the nonresonant case except for the vanishing of
the resonant component.  There are $M-1$ rather than $M$ components of
$\Xi$ to determine by minimization of $\delta W$.  The result of this
constrained minization procedure is the same as for the nonresonant
case, that the minimizing value of $\Xi$ is the one that makes the
derivatives of the nonresonant components continuous.  This allows us to
continue the minimization procedure into the next interval by
analytically continuing the nonresonant components of $\t{U}(\psi)$ and
by restarting the small nonresonant component with an arbitrary initial
value, to be determined by boundary conditions further to the right.
The proof of the necessary condition for stability follows unmodified,
using this continued matrix.  Continuation past successive singular
surfaces follows the same procedure.

An effective numerical procedure for implementing continuation past each
singular surface is based on an adaptation of Gaussian elimination
applied to the matrix of solutions.  Each independent solution vector is
dominated by its admixture of the large resonant solution on approach to
a resonant surface.  We choose one of these and subtract the multiple of
it from each of the other solution vectors which eliminates its first
resonant component.  This leaves all but the first solution vector
effectively free of the large resonant solution.  The first solution
vector is then reinitialized to the small resonant solution on the other
side of the singular surface.  Since the coefficient of this small
solution is arbitrary until it is determined by boundary conditions
further to the right, there is no loss of generality in eliminating the
second resonant component from each of the other vectors.  The
nonresonant solution vectors are then continued unmodified on the other
side of the singular surface.

The results of this section show that the equilibrium is stable to
internal modes if and only if the critical determinant $D_C$ has no
poles.  In that case, the potential energy in the plasma region is given
by
\begin{equation}
\delta W_P = \Xi^\dagger \t{W}_P \Xi,
\label{eq:plasma_energy}
\end{equation}
evaluated at the plasma-vacuum interface.  This justifies the name of
the plasma response matrix.  If the plasma is surrounded by a vacuum,
there may still be external, or free-boundary, instabilities involving
motion of the boundary.  In that case, the potential energy in the
vacuum region may be similarly expressed in terms of a vacuum response
matrix evaluated on the same surface.  The total potential energy is the
sum of the two, and the existence of free boundary modes is determined
by the lowest eigenvalue of the sum.  This is the topic of the next
section.

\section{\label{sec:free}Free-Boundary Modes}

Section \ref{sec:fixed} describes a procedure for determining the
stability of the plasma to internal modes.  If the critical determinant
$D_C$ has poles in the plasma region, then the potential energy $\delta
W$ is unbounded below and can therefore always be made negative.  If
$D_C$ has no poles in the plasma region, then $\delta W$ is bounded
below by the Hilbert invariant integral, or equivalently by the value of
$\delta W$ for the Euler-Lagrange solution constructed in Section
\ref{sec:fixed}.  If the plasma is bounded by a conducting wall, this
lower bound is zero and the plasma is therefore stable.  This is the
only case treated by Newcomb, extended to include a perfectly-conducting
pressureless plasma but not a vacuum region outside the plasma.

The purpose of this section is to extend the analysis to include
free-boundary modes, in which the plasma region is bounded by a vacuum
region rather than a conducting wall.  The plasma contribution to
$\delta W$ is given by Eq. (\ref{eq:plasma_energy}) in terms of the
plasma response matrix.  In this section we compute a vacuum response
matrix.  The total perturbed potential energy is the sum of the two.  If
a solution exists, satisfying the boundary conditions, for which the
plasma energy is negative and exceeds the positive-definite vacuum
energy, then $\delta W$ can be made negative and there is an unstable
free-boundary mode.  Much of this section is based on an extension of
the work of Chance.\cite{msc97}

If the plasma region is bounded by a separatrix, then the vacuum region
contains open field lines and cannot be treated by the methods of the
previous sections, which rely on Hamada coordinates defined only on
closed field lines.  We therefore require a treatment of the vacuum
region which does not rely on such coordinates outside the plasma
region.  However, we make use of Hamada coordinates at the plasma-vacuum
interface.  If this interface is a separatrix, we take the boundary to
be just inside the separatrix.

We begin with a brief summary of the Green's function formalism.  The
perturbed magnetic field \v{b} in the vacuum region, where $\nabla
\times \v{b} = 0$, is expressed in terms of a scalar magnetic potential,
$\v{b} = - \nabla \varphi$.  $\nabla \cdot \v{b} =0$ then implies that
$\varphi(\v{x})$ satisfies Laplace's equation, $\nabla^2 \varphi = 0$.

The scalar magnetic potential must in general be allowed to be
non-single-valued.  This follows from the integral form of Amp\'ere's
law,
\[
\oint \v{B} \cdot d\v{l} 
	= - \int_0^L \frac{\partial \varphi}{\partial l} dl
	= - \varphi \big |_0^L = \mu_0 I
\]
for any closed loop of length $L$, where $I$ is the current passing
throught the closed loop.  For example, if the loop encircles the plasma
the short way, then $I$ is the toroidal current through the plasma.
However, we use the magnetic scalar potential only for the perturbed
magnetic field \v{b} with toroidal mode number $n \ne 0$, and for this
case, there is no contribution from the total current.\cite{rcg76} Thus,
those perturbed scalar potential $\varphi$ is single-valued.

The Green's function method is used to derive an integral equation for a
solution $\varphi(\v{x})$ satisfying Laplace's equation and the boundary
conditions.  The derivation is most easily done by starting with the
more general Poisson equation,
\[
\nabla^2 \varphi(\v{x}) = -4 \pi \rho(\v{x})
\]
and then setting the source term $\rho(\v{x})$ to zero.  The Green's
function is defined as the solution to Poisson's equation with a
delta-function source,
\begin{equation}
\nabla^2 G(\v{x},\v{x}') 
	= \nabla'^2 G(\v{x},\v{x}') 
	= -4 \pi \delta(\v{x} - \v{x}'),
\label{eq:green1}
\end{equation}
which vanishes at infinity.  We define the integral
\begin{eqnarray}
{\cal I} \equiv \int_V d\v{x}' G(\v{x},\v{x}') \rho(\v{x}') \nonumber \\
= - \frac{1}{4 \pi} \int_V d\v{x}' G(\v{x},\v{x}') 
	\nabla'^2 \varphi(\v{x}'),
\end{eqnarray}
where $V$ is the volume of the vacuum region.  We now perform
two integrations by parts to transfer the derivates from $\varphi$ to
$G$, obtaining
\begin{eqnarray}
{\cal I} =&& - \frac{1}{4 \pi} \int_V d\v{x}' \nabla'^2 G(\v{x},\v{x}')
	\varphi(\v{x}') \nonumber \\
&&+ \frac{1}{4 \pi} \int_S d\v{x}' \left[ 
	\hat{\bf n} \cdot \nabla' G(\v{x},\v{x}') \varphi(\v{x}')
	\right. \nonumber \\
&&\left. - G(\v{x},\v{x}') \hat{\bf n} \cdot \nabla' \varphi(\v{x}') 
	\right]
\end{eqnarray}
where $S$ is the boundary of the vacuum region and \v{n} is the unit
outward normal to the vacuum region.  Now we use Eq. (\ref{eq:green1})
to obtain
\begin{eqnarray}
- \frac{1}{4 \pi} \int_V d\v{x}' \nabla'^2 G(\v{x},\v{x}')
	\varphi(\v{x}') \nonumber \\
= \int_V d\v{x}' \delta(\v{x}-\v{x}') \varphi(\v{x}')
= \Omega \varphi(\v{x}),
\end{eqnarray}	
where $\Omega = 1$ if \v{x} is in the vacuum region $V$, $\Omega = 1/2$
if \v{x} is on the boundary $S$ of the vacuum region, and $\Omega = 0$
if \v{x} is outside the vacuum region.  Finally, we let \v{x} be on the
boundary, let $\rho(\v{x}') = 0$, and obtain an integral equation for
$\varphi$ on the boundary,
\begin{eqnarray}
{\cal I} = \varphi(\v{x}) + \frac{1}{2 \pi} \int_S d\v{x}' \left[
	\hat{\bf n} \cdot \nabla' G(\v{x},\v{x}') \varphi(\v{x}') 
	\right. \nonumber \\
\left. - G(\v{x},\v{x}') \hat{\bf n} \cdot \nabla' \varphi(\v{x}') 
\right] = 0.
\label{eq:green2}
\end{eqnarray}

Since $\varphi$ is the scalar magnetic potential, $\hat{\bf n} \cdot
\nabla \varphi$ is the normal magnetic field and is known from boundary
conditions.  Thus, Eq. (\ref{eq:green2}) may be regarded as an equation
for $\varphi$, with the normal derivative as a known inhomogeneity.
Since the vacuum energy is the integral of the perturbed magnetic energy
density over the vacuum region, which can also be expressed as an
integral over $S$,
\begin{equation}
\delta W_V = \int_V d\v{x} |\v{b}|^2 
= \int_V d\v{x} |\nabla \varphi|^2 
= \int_S d\v{x} \varphi \hat{\v{n}} \cdot \nabla \varphi,
\label{eq:vacuum_energy}
\end{equation}
the whole vacuum problem can be expressed in terms of $\varphi$ and its
normal derivative on $S$, with no further reference to $\varphi$ inside
the vacuum region.

The surface $S$ bounding the vacuum region may consist of various
noncontiguous segments, including the vacuum-plasma interface, a
conducting wall or the surface at infinity, and various internal
metallic conductors such as passive stabilizers and active coils.  The
variables \v{x} and $\v{x}'$ vary over all of these surfaces.  Since the
normal magnetic field vanishes at the surface of a perfect conductor,
the inhomogeneity vanishes there, as it does also at infinity.
Likewise, the surface integral for $\delta W_V$ is nonvanishing only
over the plasma-vacuum interface.  The role of the external conductors
is to modify the homogeneous terms in Eq. (\ref{eq:green2}).  In the
following discussion, we consider only the simplest case, with no
external conductors and the vacuum region extending to infinity.
Extension of the treatment to include external conductors is
straightforward.

At the plasma-vacuum interface, the normal magnetic field is known in
terms of its Fourier components in Hamada coordinates.  We wish to
express the integral equation and the vacuum energy,
Eqs. (\ref{eq:green2}) and (\ref{eq:vacuum_energy}), in terms of these
Fourier components.  We therefore write
\[
\varphi(\v{x}) = \sum_{m=-\infty}^\infty \sum_{n=-\infty}^\infty 
\varphi_{m,n}(V) \exp[2 \pi i (m \theta - n \zeta)].
\]
As for the displacement vector in Eq. (\ref{eq:Xi_vector}), we define
the complex vector
\[
\Phi(V) \equiv \{ \varphi_{m,n}(V),\ m_{min} \le m \le m_{max} \} 
\]
The surface integral can be expressed in Hamada coordinates as
\[
\int_S d\v{x}' = \int_0^1 d\theta \int_0^1 d\zeta |\nabla V|. 
\]

The boundary condition at the plasma-vacuum interface is that the normal
components of the magnetic field \v{Q} in the plasma and the magnetic
field \v{b} in the vacuum must match.  Noting that the outward normal to
the vacuum is inward to the plasma, we can write $\hat{\v{n}} = -\nabla
V/|\nabla V|$.  Then the boundary condition is given by $ \v{b} \cdot
\nabla V =|\nabla V| \hat{\bf n} \cdot \nabla \varphi = \v{Q} \cdot
\nabla V = \v{B} \cdot \nabla \xi_V$.

We now multiply the integral equation in Eq. (\ref{eq:green2}) by
$\exp[-2 \pi i (m \theta - n \zeta)]$ and integrate over $\theta$ and
$\zeta$ to convert it to a matrix equation,
\begin{equation}
(\t{I} + \t{L}) \Phi
	= 2 \pi i \chi' \t{R} \t{Q} \Xi
\label{eq:int_equation}
\end{equation}
where the diagonal matrix \t{Q} is defined in Eq. (\ref{eq:define_mq})
and
\begin{widetext}
\begin{eqnarray}
L_{m,m'} &&= - \frac{1}{2 \pi} \int_0^1 d\theta \int_0^1 d\theta'
	\int_0^1 d\zeta \int_0^1 d\zeta' 
	\exp [ 2 \pi i (m' \theta' - m \theta)]
	\nabla' V \cdot \nabla' 
	G(V,\theta,\zeta;V,\theta',\zeta')
	\exp[2 \pi n i (\zeta - \zeta')], \nonumber \\
R_{m,m'} &&= - \frac{1}{2 \pi} \int_0^1 d\theta \int_0^1 d\theta'
	\int_0^1 d\zeta \int_0^1 d\zeta' 
	\exp [ 2 \pi i (m' \theta' - m \theta)]
	G(V,\theta,\zeta;V,\theta',\zeta')
	\exp[2 \pi n i (\zeta - \zeta')].
\label{eq:define_lr}
\end{eqnarray}
\end{widetext}

The toroidal straight-fieldline coordinate can be expressed as
\[
\zeta = \frac{\phi}{2\pi} + \nu(\theta,V),
\]
where $\nu(\theta,V)$ is a periodic function of $\theta$.  Using this,
the $\zeta$ integrals over the Green's function $G$ can be expressed in
terms of associated Legendre functions.  The solution to
Eq. (\ref{eq:green1}) is
\begin{eqnarray}
&&G(\v{x},\v{x}') = |\v{x}-\v{x}'|^{-1} \nonumber \\
&&= \left[ R^2 + R'^2 + (Z - Z')^2 - 2 R R' \cos \psi \right]^{-1/2}
\end{eqnarray}
where $(R,Z,\phi)$ are cylindrical coordinates and $ \psi = \phi - \phi'
$.  We define
\begin{eqnarray}
&&{\cal G}^n(\v{x},\v{x}') 
\equiv \frac{1}{2 \pi} \int_0^{2 \pi} d\psi 
	e^{-in\psi} G(\v{x},\v{x}') \nonumber \\
=&& \frac{1}{2 \pi} \int_0^{2 \pi} d\psi \cos n\psi \nonumber \\
&&\times \left[ R^2 + R'^2 + (Z - Z')^2 - 2 R R' \cos \psi \right]^{-1/2}.
\end{eqnarray}
Defining the quantities $\alpha$ and $\eta$ by
\begin{eqnarray}
\alpha^2 \cosh \eta =&& R^2 + R'^2 + (Z - Z')^2, \nonumber \\
\alpha^2 \sinh \eta =&& - 2 R R', \nonumber \\
\alpha^4 =&& \left[ R^2 + R'^2 + (Z - Z')^2 \right]^2 
	- \left( 2 R R' \right)^2 \nonumber \\
=&& \left[ (R - R')^2 + (Z - Z')^2 \right] \nonumber \\
&& \times \left[ (R + R')^2 + (Z - Z')^2 \right],
\end{eqnarray}
we can express the Green's function as$^{22}$
\begin{eqnarray}
 {\cal G}^n(\v{x},\v{x}') 
&&= \frac{1}{2 \pi \alpha} \int_0^{2 \pi} d\psi \cos n \psi \nonumber \\
&&\left[ \cosh \eta + \sinh \eta \cos \psi \right]^{-1/2} \nonumber \\
&&= \frac{(-1)^n}{\alpha \sqrt \pi} \Gamma(1/2 - n) P_{-1/2}^n(\cosh \eta)
	\nonumber \\
\end{eqnarray}
where $P_{-1/2}^n(\cosh \eta)$ is an associated Legendre function.
Finally, this allows us to simplify Eq. (\ref{eq:define_lr}) to
\begin{widetext}
\begin{eqnarray}
L_{m,m'} &&= - \frac{1}{2 \pi} \int_0^1 d\theta \int_0^1 d\theta'
	\exp [ 2 \pi i (m' \theta' - m \theta)]
	\exp \{ 2 \pi n i [\nu(\theta)- \nu(\theta')]\}
	\nabla' V \cdot \nabla' 
	{\cal G}^n (V,\theta;V,\theta'), \nonumber \\
R_{m,m'} &&= - \frac{1}{2 \pi} \int_0^1 d\theta \int_0^1 d\theta'
	\exp [ 2 \pi i (m' \theta' - m \theta)]
	\exp \{ 2 \pi n i [\nu(\theta)- \nu(\theta')]\}
	{\cal G}^n (V,\theta;V,\theta'). \nonumber
\end{eqnarray}
\end{widetext}
The remaining integrals over $\theta$ and $\theta'$ depend on the shape
of the outermost flux surface and must be performed numerically.

Once \t{L} and \t{R} are evaluated, Eq. (\ref{eq:int_equation}) can be
solved for $\Phi$ and used to evaluate $\delta W_V$,
Eq. (\ref{eq:vacuum_energy}), as
\begin{eqnarray}
\delta W_V 
&&= - \int_0^1 d\theta \int_0^1 d\zeta \varphi(V,\theta,\zeta)
	\nabla V \cdot \nabla \varphi(V,\theta,\zeta) \nonumber \\
&&= 2 \pi i \chi' \sum_{m=-\infty}^\infty (m-nq) \nonumber \\
&&\qquad [\varphi^*_{m,n}(V) \xi_{m,n}(V) 
	- \xi^*_{m,n}(V) \varphi_{m,n}(V) ] \nonumber \\
&&= 2 \pi i \chi' 
	(\Phi^\dagger \t{Q} \Xi - \Xi^\dagger \t{Q} \Phi).
\end{eqnarray}
Finally, using Eq. (\ref{eq:int_equation}), we can express this as $
\delta W_V = \Xi^\dagger \t{W}_V \Xi$, with
\[
\t{W}_V = (2 \pi \chi')^2 \t{Q} 
\left[(\t{I}+\t{L})^{-1} \t{R} 
+ \t{R}^\dagger (\t{I}+\t{L}^\dagger)^{-1}\right] \t{Q}.
\]

The total potential energy is given by
\[
\delta W = \delta W_P + \delta W_V 
	= \Xi^\dagger (\t{W}_P + \t{W}_V) \Xi 
	= \Xi^\dagger \t{W} \Xi.
\]
This is expressed in terms of a particular solution vector $\Xi$.  We
must still determine whether any linear combination of the solutions for
$\Xi$ satisfying the boundary conditions at the magnetic axis can make
$\delta W$ negative.  This is the case if and only if the lowest
eigenvalue of the total response matrix \t{W} is negative.  The
eigenvectors corresponding to the unstable eigenvalues represent the
spatial structure of the unstable modes.  This completes the
determination of stability.

\section{\label{sec:conclude}Discussion and Conclusions}

We have introduced a new method for determining the ideal MHD stability
of axisymmetric toroidal plasmas to nonaxisymmetric perturbations.  We
have generalized the Direct Criterion of Newcomb for the stability of
internal modes from his original cylindrical treatment to a full
toroidal treatment.  We have further extended the treatment to apply to
external, or free-boundary modes.

The overall procedure involves a sequence of criteria.  First we test
for Mercier stability, $D_I < 0$, Eq. (\ref{eq:di}), on all flux
surfaces.  If this criterion is violated, the equilibrium is unstable.
If it is satisfied, we then test for stability to perturbations of each
toroidal mode number $n \ne 0$.  We integrate the Euler-Lagrange
Equations from the axis to the plasma-vacuum interface, initializing it
at the axis and crossing each resonant surface according to the
procedures discussed in Section IV.  We monitor the critical determinant
$D_C$, Eq. (\ref{eq:define_dc}).  If it has a pole in the plasma region,
there is an ideal internal instability, regardless of conditions at or
beyond the plasma-vacuum interface.  If there are no poles, we compute
the plasma and vacuum contributions to the potential energy matrix
described in Section V and determine whether it has any negative
eigenvalues.  If so, there are external, or free-boundary, ideal
instabilities.  Otherwise, the system is stable to all ideal MHD
perturbations.

If the system is stable to ideal modes, there may still be singular
modes, involving more subtle effects such as plasma resistivity and
diamagnetic rotation.\cite{ahg75, ahg91} The procedure presented here
can be further extended to provide information about the behavior of the
perturbations in the ideal regions.  The asymptotic coefficients defined
in Eq. (\ref{eq:define_ca}) provide this information, generalizing the
quantity $\Delta'$ used in the cylindrical case, which must be matched
to the corresponding inner region information to determine the stablity
to singular modes.  This will be the subject of a later publication.

It would be difficult to attempt a further generalization of this
approach to nonaxisymmetric toroidal systems such as stellarators and
torsatrons, as attempted by Bineau.\cite{mb61} For an axisymmetric
system, in which the toroidal mode number is a good quantum number, the
resonant surfaces at $q=m/n$ are discrete and well-separated, except in
unusual cases in which the axis is also a resonant surface or $q'$
vanishes at a rational surface.  In a nonaxisymmetric system, since
there are in principle infinitely many toroidal and poloidal harmonics,
the singular surfaces would be dense everywhere, and this would appear
to make the procedure impractical.  Perhaps it could be shown that
convergence is achieved for a moderate number of toroidal harmonics.
Even then, the situation would be further complicated by the existence
of stochastic regions, especially in the neighborhood of low-order
resonant surfaces, as occur in most nonaxisymmetric configurations,
where flux surfaces, and hence flux coordinates, fail to exist.

\begin{acknowledgments}
The author acknowledges extensive discussions with Leonid Zakharov,
Morrell Chance, and Alexander Popov.  He has also benefited from many
useful suggestions from Dan Barnes, Steve Cowley, John Finn, Richard
Fitzpatrick, Dick Gerwin, Sheryl Glasser, John Greene, Steve Jardin,
Chuck Kessel, J. Manickam, Stefano Migliuolo, Bob Miller, Phil Morrison,
Rick Nebel, Bill Newcomb, Don Pearlstein, Martin Peng, Jesus Ramos,
Walter Sadowski, and Alan Turnbull.  This work was supported by a
contract from the U. S.  Department of Energy, Office of Fusion Energy.
\end{acknowledgments}

\appendix

\section{\label{sec:coef}Coefficient Matrices}

In this appendix we give detailed expressions for the coefficient
matrices \t{A}, \t{B}, \t{C}, \t{D}, \t{E}, \t{H}, \t{F}, \t{G}, and
\t{K}.

Before expressing these principal matrices, we define some more
fundamental matrices in which are used in their definition.  We define
the following scalar components of the metric tensor:
\begin{eqnarray}
g_{11} &&\equiv |\nabla \theta \times \nabla \zeta|^2, \nonumber \\
g_{22} &&\equiv |\nabla \zeta \times \nabla \psi|^2, \nonumber \\
g_{33} &&\equiv |\nabla \psi \times \nabla \theta|^2, \nonumber \\
g_{23} &&\equiv (\nabla \zeta \times \nabla \psi)
	\cdot (\nabla \psi \times \nabla \theta), \nonumber \\
g_{31} &&\equiv (\nabla \psi \times \nabla \theta)
	\cdot (\nabla \theta \times \nabla \zeta), \nonumber \\
g_{12} &&\equiv (\nabla \theta \times \nabla \zeta)
	\cdot (\nabla \zeta \times \nabla \psi).
\end{eqnarray}
For any scalar $f(\theta)$, we define an associated matrix $<f>$ as the
matrix of Fourier coefficients,
\[
<{\cal F}>_{mm'} \equiv \int_0^1 d\theta {{\cal J}}
	\exp[2 \pi i (m'-m) \theta]  {\cal F}(\theta) 
\]
Note that if $f$ is real, then $<f>$ is Hermitian.  The following
quantities are required when the Jacobian is not uniform:
\[
J_{mm'} \equiv <1>_{mm'}, \qquad
J'_{mm'} \equiv <{\cal J}'/{\cal J}>_{mm'}
\]
Next we define the matrices
\[
(G_{ij})_{mm'} \equiv <g_{ij}>_{mm'}
\]
Finally, we define the diagonal matrices \t{M} and \t{Q} in terms of
their diagonal components,
\begin{equation}
M_{m,m'} \equiv m \delta_{m,m'}, \qquad 
Q_{m,m'} \equiv (m - n q) \delta_{m,m'}.
\label{eq:define_mq}
\end{equation}

In terms of these fundamental matrices, the coefficient matrices
appearing in Eq. (\ref{eq:dw3}) are given by
\begin{widetext}
\begin{eqnarray}
\t{A} =&& (2 \pi)^2 \left[ n (n \t{G}_{22} + \t{G}_{23} \t{M}
	+\t{M} (n \t{G}_{23} + \t{G}_{33} \t{M})  \right] \nonumber \\
\t{B} =&& -2 \pi i \chi' \left[ n (\t{G}_{22} + q \t{G}_{23} )
	+ \t{M} (\t{G}_{23} + q \t{G}_{33})  \right] \nonumber \\
\t{C} =&& -2 \pi i \left[  \chi'' (n \t{G}_{22} + \t{M} \t{G}_{23})
	+ (q \chi')' (n \t{G}_{23} + \t{M} \t{G}_{33})  \right] \nonumber \\
	&& - (2 \pi)^2 \chi' (n \t{G}_{12} + \t{M} \t{G}_{31}) \t{Q}
	+ 2 \pi i (2 \pi f' \t{Q} - n P'/\chi' \t{J}) \nonumber \\
\t{D} =&& \chi'^2 \left[ (\t{G}_{22} + q \t{G}_{23})
	+ q(\t{G}_{23} + q \t{G}_{33}) \right] \nonumber \\
\t{E} =&& \chi' \left[  \chi'' (\t{G}_{22} + q \t{G}_{23})
	+ (q \chi')' (\t{G}_{23} + q \t{G}_{33}) \right] \nonumber \\
	&& - 2 \pi i \chi'^2 (\t{G}_{12} + q \t{G}_{31}) \t{Q}
	+ P' \t{J} \nonumber \\
\t{H} =&& \chi''\left[ \chi'' \t{G}_{22} + (q \chi')' \t{G}_{23} \right]
	+ (q \chi')'\left[ \chi'' \t{G}_{23} 
	+ (q \chi')' \t{G}_{33} \right] \nonumber \\
	&& + 2 \pi i \chi' \left[ \chi'' (\t{M} \t{G}_{12} - \t{G}_{12} \t{M})
	+ (q \chi')' (\t{M} \t{G}_{31} - \t{G}_{31} \t{M})  \right] \nonumber \\
	&& + (2 \pi \chi')^2 \t{Q} \t{G}_{11} \t{Q}
	+ \left[ P' \left(  \t{J} \chi''/\chi' + \t{J}' \right) 
	-2 \pi f' q' \chi' \t{I} \right].
\label{eq:define_a-h}
\end{eqnarray}
\end{widetext}
Note that all of the off-diagonal components of these matrices arise
from the metric tensor quantities $\t{G}_{ij}$, which in turn arise from
the stabilizing $Q^2$ term in Eq. (\ref{eq:dw1}), and thus all
potentially destabilizing terms in $\delta W$ are diagonal.  This is a
useful property of Hamada coordinates.  Note also that the matrices
\t{A}, \t{D}, and \t{H}, which couple like quantities in
Eq. (\ref{eq:dw3}), are Hermitian, while \t{B}, \t{C}, and \t{E}, which
couple unlike quantities, are not.

Two of the composite matrices defined in Eq. (\ref{eq:define_fkg})
contain important cancellations.  Thus, we can express
\begin{eqnarray}
\t{F} \equiv&& \t{D} - \t{B}^\dagger \t{A}^{-1} \t{B} \nonumber \\
=&& (\chi' / n)^2 \t{Q} \left[ \t{G}_{33} 
	- (n \t{G}_{23} + \t{G}_{33} \t{M})  \right. \nonumber \\
&&\left. \times
	(2 \pi)^2 \t{A}^{-1}
	(n \t{G}_{23} +  \t{M} \t{G}_{33}) \right] \t{Q}
\label{eq:simplify_f}
\end{eqnarray}
Note the presence of the factor \t{Q} on either side of \t{F}.  This is
responsible for the vanishing of $\det \t{F}$ at the resonant surfaces,
discussed in detail in Section III.  Similarly, we can express
\begin{eqnarray}
\t{K} =&& \t{E} - \t{B}^\dagger \t{A}^{-1} \t{C} \nonumber \\
	=&& \t{Q} (\chi'/n) \left\{ 2 \pi i 
	(n \t{G}_{23} + \t{G}_{33} \t{M}) \t{A}^{-1} \t{C} 
	\right. \nonumber \\
&&\left. - [ \chi'' \t{G}_{23} + (q \chi')' \t{G}_{33}
	- 2 \pi i \chi' \t{G}_{31} \t{Q} + J_\theta \t{I}]
	\right\} \nonumber \\
\label{eq:simplify_k}
\end{eqnarray}
No such simplification has been found for the remaining matrix \t{G}
defined in Eq. (\ref{eq:define_fkg}).

\bibliography{pub1}
\end{document}
