%\setlength{\parskip}{1.ex}
The magnetic field $\nablav\chi(\theta,\zeta)$ in the vacuum region is
driven by the normal component of the perturbed magnetic field at the
plasma-vacuum interface. It naturally enters the equations in the
\textsc{vacuum} code in the form
%
\beq
\mathcal{B}(\theta,\zeta) & = & \mathcal{J}\nablav \chi\cdot\nablav\mathcal{Z},\\
            & = & - Z_\theta \frac{\partial\chi}{\partial X} +
            X_\theta\frac{\partial \chi}{\partial Z},
\eeq
%
so that, given the parametrization, $[X(\theta), Z(\theta)]$,
$\mathcal{B}$ doesn't involve the Jacobian directly and the Neumann
problem is independent of the coordinate used.

Fourier analysis can be carried out in any $(\theta,\zeta)$ system:
%
\beq
\mathcal{B}(\theta,\zeta) & = & \sum_l b_l e^{i(l\theta-n\zeta)} 
\eeq
%
and $\chi$ is calculated as a response matrix to the vector containing
$b_l$.

If $\xi_\psi(\theta,\zeta)$ is supplied as the driving term, then in any
$\theta$,
%
\beq
\mathcal{B}(\theta,\zeta) & = & \mathcal{J} \mathbf{Q} \cdot\nablav\mathcal{Z} \\
  & = & \mathcal{J}\mathbf{B}\cdot\nablav\xi_\psi(\theta,\zeta)\\ & = &
  \mathcal{J}
  \mathbf{B}\cdot\nablav\theta\left[\frac{\partial}{\partial \theta} -
  i\frac{\mathbf{B}\cdot\nablav\zeta}{\mathbf{B}\cdot\nablav\theta} n
  \right]\xi_\psi(\theta,\zeta),
\eeq
%
so information about both $\mathbf{B}$ and $\mathcal{J}$ is needed.

However, if the coordinates are such that the field lines are straight
then the relation,
%
\beq 
b_l & \sim & i(l-nq)\xi_l, \label{eqn:bl}
\eeq
%
holds, and then the $\theta$ must be chosen to be the \textsc{pest}
angle, $\theta_\mathrm{p}$ if the toroidal angle is $\phi$.

Alternatively, for an arbitrary choice of $\theta$, we require the
parametrization of $X(\theta), Z(\theta)$, and the coordinate
transformation contained in $\nu(\theta)$, where $\zeta = \phi +
\nu(\theta)$. Then the Fourier coefficients of $\mathcal{B}$ can be
written as \eqref{eqn:bl}.
