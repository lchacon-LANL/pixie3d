\documentclass[10pt]{article}
\usepackage{ltexe_com}
\textw{7.0in}
\texth{9.0in}
%
\newcommand{\styla}[1]{\textit{#1}}
\newcommand{\stylb}[1]{\textbf{#1}}
\newcommand{\stylc}[1]{\textsc{#1}}
\newcommand{\styld}[1]{\textit{#1}}
\newcommand{\style}[1]{\texttt{#1}}
\newcommand{\nablav}{\mbox{\boldmath{$\nabla$}}}
\newcommand{\notr}{\textsl{not relevant}}
\renewcommand{\qedge}{q_\mathrm{edge}}
\newcommand{\prad}{p_{\mathrm{rad}}}
\newcommand{\xma}{X_{\text{cen}}}
\newcommand{\zma}{Z_{\text{cen}}}
\newcommand{\Xp}{X_{p_i}}
\newcommand{\Xw}{X_{w_i}}
\newcommand{\Zp}{Z_{p_i}}
\newcommand{\Zw}{Z_{w_i}}
%
\setlength{\parskip}{1.ex}
%
\begin{document}
%
%\draft
%
%\Large
\begin{center}
{\large\textsf{DRAFT}}\\[1.ex]
\textbf{Input Documentation for the {\small VACUUM} Code}\\
%
M. S. Chance\\

\textit{Princeton University Plasma Physics Laboratory,\\
P.O. Box 451, Princeton, NJ 08543, USA
}\\
%
%\date{\today}
Draft dated \today
%
\end{center}
%
%\maketitle
%
\section{Preliminaries}

Because of historical reasons and the continual addition of features
and refinements, the look and feel of the data may not seem polished
and streamlined. It will someday. Thus, some variables are redundant
and some of the control variables [logical and integers] may seem to
overlap. In general, it's best to turn off switches which are not
relevant to your case.

The code has been used most extensively for up-down symmetric
configurations and so most confidence in the results are for this
symmetry. It has been used for simple applications for non up-down
symmetric configurations but some of the ancillary applications (like
those in the the diagnostic capabilities) have not been thoroughly
checked out as of now.

\begin{itemize}

\item Radius means along the horizontal. 

\item The following are excerpted from the published paper.  The
latter contains more details about the various shell options and the
Mirnov loop and eddy current diagnostic capabilities.

\begin{itemize}

   \item
    Some definitions used here are the
   plasma `radius', $\prad$ and the plasma center $(X_{\mathrm{cen}},
   Z_{\mathrm{cen}})$, which are defined by:
   %
   \beq
   \prad  & \equiv & \oover{2}( X_{\mathrm{max}} - X_{\mathrm{min}} ) \\
     X_{\mathrm{cen}} & = & \oover{2} ( X_{\mathrm{max}} + X_{\mathrm{min}} )
   \eeq
   and
\beq
  Z_{\mathrm{cen}} & = & \oover{2} ( Z_{\mathrm{ min}} + Z_{\mathrm{max}} ),
\eeq
where $X_{\mathrm{min}}$ and $X_{\mathrm{max}}$ are respectively, the minimum
and the maximum $X$-values of the plasma, etc.

   \item There is an option in the code which numerically
redistributes the grid along the shell so that the grid is equidistant
between adjacent points. While this usually gives the best overall
accuracy, this option should not be exercised when the wall is very
close to the plasma since the plasma-wall points may not be aligned
with respect the indexing. In this case, it is best to choose the
equidistant-shell option, (\style{ishape} = 6), with the equal-arc
option for the shell turned off, (\style{leqarcw} = 0).

   \item The definitions of \textit{elongation} and
\textit{triangularity} can be ascertained from the following
description of the \textbf{Simple toroidal {\em D}-shaped shell}:

This configuration where the shell is centered at $c_w$, has an
elongation factor of $b_w$, a triangularity skewness $d_w$, and a
radius in the equatorial plane of $a$, is given by the relation,
\beq
\Xw & = & c_w + a \cos ( \theta_i + d_w \sin \theta_i ) \\
\Zw & = & \mbox{}- b_w a \sin \theta_i.
\eeq
This gives a free-standing shell independent of the plasma position or
dimensions, and is useful if the shell is from the simulation of an
actual device. 

   \item An example of the parameterization of the shell which is
scaled to the plasma is given by,
%
\beq
\Xw  = && X_{\mathrm{cen}} + \prad c_w + \prad ( 1 + a - c_w ) 
\cos(\theta_i + d_w \sin \theta_i) \\
\Zw  = && \mbox{}- b_w \prad ( 1 + a - c_w ) \sin \theta_i.
\eeq
Here the center of the shell is offset by $\prad c_w$ in the equatorial
plane and is still a distance $\prad a$ from the outer major radius
side of the plasma. 

\end{itemize}

\end{itemize}

      \section{Notes on the Normalization of $\delta W$ in {\small
      VACUUM} }
%
%
The \textsc{vacuum} code calculates the response to 
%
\beq
\mathcal{B} & = & \mathcal{J}\nablav \chi\cdot\nablav\mathcal{Z},
\eeq
with
\beq
\mathcal{J} & = &
(\nablav\mathcal{Z}\cdot\nablav\theta\times\nablav\phi)^{-1},
\eeq
%
$\mathcal{Z}$ the radial coordinate and $\chi$ the magnetic
scalar potential. 

At the interface 
%
\beq
\mathcal{B} & = & \mathcal{J}\mathbf{Q}\cdot\nablav\mathcal{Z} \\
            & = & \mathcal{J}\nablav\mathcal{Z}\cdot\nablav \times
(\mathbf{\bxi\times B}) \\
            & = & \mathcal{J}\mathbf{B}\cdot\nablav
(\mathbf{\bxi}\cdot\nablav\mathcal{Z}).
\eeq

Now the equilibrium field is 
\beq
\mathbf{B} & = & \nablav\phi\times\nablav\psi + Rg\nablav\phi\\
       & = & \psi'\nablav\phi\times\nablav\mathcal{Z} + Rg\nablav\phi,
\eeq
where $\psi = \Psi/2\pi$, $\Psi$ being the poloidal flux. So
\beq
\mathbf{B}\cdot\nablav\theta & = & \mathcal{J}^{-1} \psi'
\eeq
where the prime denotes derivative with respect to $\mathcal{Z}$. Then
we have
\beq
\mathcal{B} & = & \psi'(m-nq)\mathbf{\bxi}\cdot\nablav\mathcal{Z} \label{eqn:mnq}
\eeq
in a straight-field-line coordinate system.

In the cylinder with $r$ as the radial coordinate,
\beq 
\psi' & = & R B_\theta\\
      & = & \frac{r B_z }{q},
\eeq
%
whereas if the radial coordinate is chosen to be $\psi$, then $\psi' =
1$, and the ratio between these two cases is $r B_z/q$. This is equal
to $1/q$ if $r=1$ and $B_z$ is unity.

It should be noted that \textsc{vacuum} calculates the response to
$\mathcal{B}$ but oftentimes outputs the vacuum matrix $V_{mm'}$ as
the response to $\bxi$, in which case the factor of $(m-nq)(m'-nq)$
arising from Eqn. \ref{eqn:mnq} is incorporated in $V_{mm'}$, where
$m=m'$ in the circular cylinder.

%
      \section{Form of the Source Term}
%
%\setlength{\parskip}{1.ex}
The magnetic field $\nablav\chi(\theta,\zeta)$ in the vacuum region is
driven by the normal component of the perturbed magnetic field at the
plasma-vacuum interface. It naturally enters the equations in the
\textsc{vacuum} code in the form
%
\beq
\mathcal{B}(\theta,\zeta) & = & \mathcal{J}\nablav \chi\cdot\nablav\mathcal{Z},\\
            & = & - Z_\theta \frac{\partial\chi}{\partial X} +
            X_\theta\frac{\partial \chi}{\partial Z},
\eeq
%
so that, given the parametrization, $[X(\theta), Z(\theta)]$,
$\mathcal{B}$ doesn't involve the Jacobian directly and the Neumann
problem is independent of the coordinate used.

Fourier analysis can be carried out in any $(\theta,\zeta)$ system:
%
\beq
\mathcal{B}(\theta,\zeta) & = & \sum_l b_l e^{i(l\theta-n\zeta)} 
\eeq
%
and $\chi$ is calculated as a response matrix to the vector containing
$b_l$.

If $\xi_\psi(\theta,\zeta)$ is supplied as the driving term, then in any
$\theta$,
%
\beq
\mathcal{B}(\theta,\zeta) & = & \mathcal{J} \mathbf{Q} \cdot\nablav\mathcal{Z} \\
  & = & \mathcal{J}\mathbf{B}\cdot\nablav\xi_\psi(\theta,\zeta)\\ & = &
  \mathcal{J}
  \mathbf{B}\cdot\nablav\theta\left[\frac{\partial}{\partial \theta} -
  i\frac{\mathbf{B}\cdot\nablav\zeta}{\mathbf{B}\cdot\nablav\theta} n
  \right]\xi_\psi(\theta,\zeta),
\eeq
%
so information about both $\mathbf{B}$ and $\mathcal{J}$ is needed.

However, if the coordinates are such that the field lines are straight
then the relation,
%
\beq 
b_l & \sim & i(l-nq)\xi_l, \label{eqn:bl}
\eeq
%
holds, and then the $\theta$ must be chosen to be the \textsc{pest}
angle, $\theta_\mathrm{p}$ if the toroidal angle is $\phi$.

Alternatively, for an arbitrary choice of $\theta$, we require the
parametrization of $X(\theta), Z(\theta)$, and the coordinate
transformation contained in $\nu(\theta)$, where $\zeta = \phi +
\nu(\theta)$. Then the Fourier coefficients of $\mathcal{B}$ can be
written as \eqref{eqn:bl}.

%
	\section{Input data}
\label{sec:inpdata}
The control input file is called \stylb{modivmc}. [A variable input
file on the command line can be added later].  The names of the input
data files needed from the various stability codes and the output
files are given in the tables. The notation used corresponds more or
less to the coding.

 In the first column of the tables is an explicit list of the input
variables. The second column shows the equivalent variables used in
expressions in the published paper, [\textit{Physics of Plasmas},
\textbf{4}, June 1997, 2161]. The third column gives the type of
variable and values while the fourth column gives a description of the
variables. 
 
The subsections describing the input variables are labeled like the
namelist blocks of the input file. Again note that these names are not
completely correlated with the kind of variables so the functionality
of some of the variables may span the two or more blocks. This will be
fixed soon.

\newpage

\subsection{modes}

\begin{center}
%\begin{table}[h]
\begin{tabular}{|l|l|p{1.0in}|p{3.8in}|}
\hline
Variable & Paper  & Value or &  Definition \\ 
         & equiv. &  type  & \\ \hline

lzio     &  & 1 & Reads from ``mp0'' and ``mp1'' for \stylc{pest}-type
inputs. \\
         &   & 0 & Turns off references to ``mp0'' and ``mp1''\\

mp0, mp1  &       &File names&  Plasma input from \stylc{pest}-type codes. \\

mth & & Even integer &Number of grid points used for the
                     calculation. The values of the needed quantities
                     on these points are interpolated from the those
                     gotten from the plasma information in `mp0',
                     `mp1', \stylb{vacin}, \stylb{invacuum}, or
                     equivalent.\\

lmin, lmax &      & arrays & Range of Fourier harmonics. Usually the
                    same as that used in the plasma calculation. Not
                    relevant for \stylc{dcon}. In \stylc{gato}, and
                    for the nonzero \style {lspark} options these are
                    used for internal Fourier matrix operations.\\ 

lnint, lxint&      & Integers & Range of the internal Fourier
                     harmonics used for the \style{lspark} and
                     \style{lgato = 2} options. Best if \style{lnint =
                     -lxint}. \\

mfel     &        & & Number of finite elements when the \style{lgato}
                      option is turned on. \\

lgaus    &        & integer & Needed for the \style{lgato = 3} option. The
                    first digit[3, or 5] and second digit[4, or 6] is
                    the order of the Gaussian integration over the
                    observer and source elements respectively.\\

m         &       & & \notr.\\

mdiv      &       & & \notr.\\

n         & $n$   & & The toroidal mode number. Overwritten by input from
                      \stylc{dcon} or \stylc{gato} \\

lsymz     &       & .true. & Symmetrizes the vacuum matrix\\

lfunint   &       & logical & \notr. \\

xiin      &  & array & Input Fourier modes of $\xi_l(edge)$. See
                       \style{ieig} in Sec. \ref{sec:diags}\\

leqarcw  &  & 1 & Turns on equal arcs distribution of the nodes on the
                 shell. Best results unless the wall is very close to
                 the plasma.  See \style{ishape = 6} option. \\

ladj     &  & 1 & Read input \stylb{inadjv} from and interfaces to the
                 \stylc{adj} code. Outputs \stylb{vacadj}\\
         &  & 0 & Otherwise \\

ldcon    &  &1 & Reads input from \stylc{dcon}'s \stylb{invacuum} and outputs
\stylb{vacdcon} \\

lgato    &  & $> 0$ & Calculates finite element representation of and
outputs the vacuum matrix, \stylb{vacmat}?, for the \stylc{gato}\ code. \\
         &  & 1 & Direct finite element representation with collocation. \\
         &  & 2 & Transformation to finite elements from Fourier
representation. \\
         &  & 3 & Direct finite element representation without
collocation. \\

lrgato   &  & 1 & Reads file \stylb{vacin} for input for the
\stylc{gato} code.\\

lspark & & $> 0$ & \stylc{spark} and feedback type calculations. This
                    is work in progress. Set it to 0.\\
         &  & 1 &   \\
         &  & 2 &   \\

ismth    &  & 1 & Smooths some quantities  for plotting. \\

lnova    &  & .true.  & Interfacing to the \stylc{nova} code. Takes
\stylc{pest}-type input. Outputs \stylb{vacout}\\

lpest1 & & .true.  & Interfacing to the \stylc{pest1} code\\ 

       & & .false. & Interfacing to the \stylc{pest2} code (the
         default) and all other codes but
         \stylc{pest1}. \stylc{pest}-type outputs are in
         \stylb{vacmat}\\ \hline
\end{tabular}
%\end{table}
\end{center}

\subsection{debugs}

\begin{center}
%\begin{table}[h]
\begin{tabular}{|l|l|p{1.0in}|p{3.8in}|}
\hline
Variable & Paper  & Value or &  Definition \\ 
         & equiv. &  type  & \\ \hline

checkd & & logical & Switch to: write out matrix elements. Fourier
                     analyze matrix elements and write out the
                     Fourier matrix.  \\

checke & & logical & Switch to turn on the calculation of and to write
                      out the eigenvalues and eigenfunctions of
                      selected matrices. Plots $\chi$ on the wall.\\

check1 & & logical &  Turns on timer at selected points in code. It's
                      functional only for the C-90.\\

check2 & & logical & Turns on write statements for some quantities for
                      debugging. $\delta(\theta)$, wall coordinates
                      and their derivatives, \style{xpass, zpass, ell,
                      thgr, xgrd, zgrd} and usually indices, sums
                      etc.\\

checks & & logical & Switches for checking quantities in the
                     \textbf{spark} module.\\

lkplt & & 1 & Turns on eddy current plots. Also plots plasma and wall
               coordinates. \\

wall  &  &.false.  &  Leave this .false. [Not used??]\\
\hline
\end{tabular}
%\end{table}
\end{center}

\subsection{vacdat}

\begin{center}
%\begin{table}[h]
\begin{tabular}{|l|l|p{1.0in}|p{3.8in}|}
\hline

Variable & Paper  & Value or &  Definition \\ 
         & equiv. &  type  & \\ \hline

ishape    &       & integer & Options for the wall shape. \\
          & &$< 0$   & \textbf{Spherical topogy}. \\
          & & $< 10 $  & \textbf{Closed toroidal topology}. \\
          &   & Consult the paper for more  details:\\

          & & 2 & Elliptical shell confocal to the plasma's
          radius and height. The radius of the shell is \style{a}.\\

          & & 4 & Modified dee-shaped wall independent of plasma
          gerometry with triangularity, \style{dw}, squareness(?) and 2nd
          harmonic of zwall in \style{aw} and \style{tw}. Centered at
          \style{cw}, radius \style{a} and elongation \style{bw}.\\

        &   & 5 & dee-shaped wall scaled to the radius and
          geometric center of the plalsma. Offset of \style{cw}. Other
          variables as option 4. \\

          &       & 6  & Conforming shell at distance
          \style{a}*$p_{rad}$. This is best option for a close fitting shell
          since the plasma and shell nodes are aligned. \\

          &   & 7 & Enclosing bean-shaped wall. Consult paper.\\

          &   & 8 & Wall of \textsc{diii-d}. \\ 

          & & $< 20$  & \textbf{Solid conductors not linking plasma}.\\

          &  & 11  & dee-shaped conductor.\\

          &  & 12  & Solid bean-shaped conductor on right.\\

          &  & 13  & Solid bean-shaped conductor on left.\\

          & & $< 30$ & Toroidal conductor with a toroidally symmetric
          gap.  Geometry correlated to plasma position and geometry. \\

          &       & 21 & shell scaled to plasma. Gap on the inner side.\\
          &       & 24 & shell scaled to plasma. Gap on the outer  side.\\

          & & $< 40$  & Toroidal conductor with a toroidally symmetric
          gap.  Geometry independent of plasma. \\
          
          &       & 31 & shell independent of plasma. Gap on the inner
side.\\
          &       & 34 & shell independent of plasma. Gap on the outer
side.\\

aw        & $a_w$ &  & Half-thickness of the shell.\\

bw        & $b_w$  & & Elongation of the shell.\\

cw        & $c_w$   && Offset of the center of the shell from the major
radius, $X_{\rm maj}$.\\

dw        & $\delta_w$& & Triangularity of shell.\\

tw        & $\tau_w$  && Sharpness of the corners of the shell. Try
.05 as a good initial value.\\

nsing & & & \notr \\

epsq  & &  & \notr \\

noutv & & & Number of grid points for the eddy current plots.\\

delg &  &noninteger & Size of arrows for the eddy currnet
plots. Integer part is length of shaft and decimal part is size of the
head. \\

idgt & & integer & \notr\ now. Used to be approx. number of digits
accuracy in the Gaussian elimination used in the calculation. A value
of idgt=6 is usually sufficient.\\

delfac & & & Controls grid size to calculate derivatives in
\stylb{spark} type calculations. \\

idsk & & & \notr.\\

cn0 & & & Constant added to the $\cal K$ matrix to make it nonsingular for
$n=0$ modes. \\
\hline
\end{tabular}
%\end{table}
\end{center}

\subsection{shape}

\begin{center}
%\begin{table}[h]
\begin{tabular}{|l|l|p{1.0in}|p{3.8in}|}
\hline

Variable & Paper  & Value or &  Definition \\ 
         & equiv. &  type  & \\ \hline

ipshp    &  & 0 & Get the plasma boundary and safety factor,
                 $\qedge$ etc. from input files.\\

         & & 1 & Ignores input data files. Sets $\qedge = $
         \style{qain}. Shape of plasma is dee-shaped centered at
         \style{xpl}, radius \style{apl}, elongation \style{bpl} and
         triangularity \style{dpl}. The straight-line coordinate
         variable $\delta(\theta)$ is set equal to  zero. \\

xpl    & & & \\
apl    &&&   \\
bpl    &&&   \\
dpl    &&&   \\

qain  & & & Input value for the $\qedge$ when \style{ipshp} = 1. \\ 

r    &&& \notr \\

a & $a$ & &Usually the distance of the shell from the plasma in units
of the plasma radius $p_{\rm rad}$ at the outer side. See
\style{ishape = 4} and \style{ishape = 2} for exceptions. If $a \ge
10$, the wall is assumed to be at $\infty$.\\ \hline

\multicolumn{4}{|c|}{\textbf{These variables are for shells with toroidal gaps}}\\

b         & $\beta$   && For  Subtending half-angle of the shell in degrees.\\

abulg     & $a_b$    & & The size of the bulge along the major radius
in normalized to the mean plasma radius. \\

bbulg     & $\beta_b$ & & Subtending half-angle of the extent of the

bulge. \\

tbulg     & $\tau_b$ & & Inverse roundedness of the bulge corners.  \\ \hline
\end{tabular} \vspace{.3in}
%\end{table}
\end{center}


\subsection{diags}
\label{sec:diags}

\begin{center}
%\begin{table}[h]
\begin{tabular}{|l|l|p{1.0in}|p{3.8in}|}
\hline
Variable & Paper  & Value or &  Definition \\ 
         & equiv. &  type  & \\ \hline

lkdis & & logical & Turns on the eddy current calculations. It calls
                    \style{subroutine kdis}. \\

ieig  & & integer & Options for getting the surface eigenfunctions $\xi(l)$. \\
      & & 1 & From \stylc{pest-1}. Writes the $\omega^2$ and $\xi(l)$.\\

      & & 4 & Reads from file \stylb{outdist}. \\

      & & 5 & Gets $\xi(l)$ from the input \style{xiin} in namelist
              \style{modes}. \\
      & & 8 & $\Re[\xi(k)]$ and $\Im[\xi(k)]$ from input file
              \stylb{vacin} for \stylc{gato}'s input. \\

iloop & & integer & Turns on Mirnov coils calculation. \\
     
      & & 1 & Coils location given by dee-shaped geometry set by
      parameters given below. \\
      
      & & 2 & \stylc{pbx}'s Mirnov coil positions. \\

x,zloop & & & Input positions of Mirnov loops when \style{lpsub = 1} \\

nloop & & & number of coils around the plasma. \\

nloopr  & & & numner of radial loops.  \\

lpsub & & integer & Uses subroutiones for coil positions. Otherwise
uses namelist inputs \style{(xloop,zloop)}. \\

nphil & & & number of $\phi$ positions for the loop calculations. \\

nphse & & & number of $\phi$ positions for the eddy current plots. \\

nph & & & number of such contours in $\phi$. \\

mx, mz & & & Contour grid for $\chi$. \\

xofsl & & & offset of the loop positions. \\
aloop  & & & distance of the loop ``dee'' from plasma. \\
bloop  & & & elongation of the loop ``dee''. \\
dloop  & & & triangularity of the loop ``dee''. \\
rloop  & & & \notr .\\
ntloop & & & number of loop positions distributed along the shell. \\
deloop & & & delta fraction of \style{plrad} to calculate
             magnetic field from the derivative of $\chi$.\\

\hline
\end{tabular}
%\end{table}
\end{center}


\subsection{sprk}

\begin{center}
%\begin{table}[h]
\begin{tabular}{|l|l|p{1.0in}|p{3.8in}|}
\hline
Variable & Paper  & Value or &  Definition \\ 
         & equiv. &  type  & \\ \hline

\multicolumn{4}{|c|}{\textbf{Spark and Feedback type variables. Under Development}}\\

\hline
\end{tabular}
%\end{table}
\end{center}

For the plasma-independent equivalent, substitute the followling
definitions. The rest have the same meaning as above but not 
scaled to the plasma.\vspace{.1in}

\begin{center}
%\begin{table}[h]
\begin{tabular}{|l|l|p{1.0in}|p{3.8in}|}
\hline

Variable & Paper  & Value or &  Definition \\ 
         & equiv. &  type  & \\ \hline

cw       & $c_w$ & & Center of the shell.       \\

a        & $a$   & & Inner radius of the shell. \\ \hline
\end{tabular} \vspace{.1in}
%\end{table}
\end{center}
%

\end{document}
