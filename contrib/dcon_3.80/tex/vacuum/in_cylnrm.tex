%
The \textsc{vacuum} code calculates the response to 
%
\beq
\mathcal{B} & = & \mathcal{J}\nablav \chi\cdot\nablav\mathcal{Z},
\eeq
with
\beq
\mathcal{J} & = &
(\nablav\mathcal{Z}\cdot\nablav\theta\times\nablav\phi)^{-1},
\eeq
%
$\mathcal{Z}$ the radial coordinate and $\chi$ the magnetic
scalar potential. 

At the interface 
%
\beq
\mathcal{B} & = & \mathcal{J}\mathbf{Q}\cdot\nablav\mathcal{Z} \\
            & = & \mathcal{J}\nablav\mathcal{Z}\cdot\nablav \times
(\mathbf{\bxi\times B}) \\
            & = & \mathcal{J}\mathbf{B}\cdot\nablav
(\mathbf{\bxi}\cdot\nablav\mathcal{Z}).
\eeq

Now the equilibrium field is 
\beq
\mathbf{B} & = & \nablav\phi\times\nablav\psi + Rg\nablav\phi\\
       & = & \psi'\nablav\phi\times\nablav\mathcal{Z} + Rg\nablav\phi,
\eeq
where $\psi = \Psi/2\pi$, $\Psi$ being the poloidal flux. So
\beq
\mathbf{B}\cdot\nablav\theta & = & \mathcal{J}^{-1} \psi'
\eeq
where the prime denotes derivative with respect to $\mathcal{Z}$. Then
we have
\beq
\mathcal{B} & = & \psi'(m-nq)\mathbf{\bxi}\cdot\nablav\mathcal{Z} \label{eqn:mnq}
\eeq
in a straight-field-line coordinate system.

In the cylinder with $r$ as the radial coordinate,
\beq 
\psi' & = & R B_\theta\\
      & = & \frac{r B_z }{q},
\eeq
%
whereas if the radial coordinate is chosen to be $\psi$, then $\psi' =
1$, and the ratio between these two cases is $r B_z/q$. This is equal
to $1/q$ if $r=1$ and $B_z$ is unity.

It should be noted that \textsc{vacuum} calculates the response to
$\mathcal{B}$ but oftentimes outputs the vacuum matrix $V_{mm'}$ as
the response to $\bxi$, in which case the factor of $(m-nq)(m'-nq)$
arising from Eqn. \ref{eqn:mnq} is incorporated in $V_{mm'}$, where
$m=m'$ in the circular cylinder.
